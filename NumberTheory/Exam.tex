\documentclass{article}

\usepackage{fancyhdr}
\usepackage{extramarks}
\usepackage{amsmath}
\usepackage{amsthm}
\usepackage{amsfonts}
\usepackage{tikz}
\usepackage[plain]{algorithm}
\usepackage{algpseudocode}
\usepackage{mathtools}

\usepackage[utf8]{inputenc}

% Default fixed font does not support bold face
\DeclareFixedFont{\ttb}{T1}{txtt}{bx}{n}{12} % for bold
\DeclareFixedFont{\ttm}{T1}{txtt}{m}{n}{12}  % for normal

% Custom colors
\usepackage{color}
\definecolor{deepblue}{rgb}{0,0,0.5}
\definecolor{deepred}{rgb}{0.6,0,0}
\definecolor{deepgreen}{rgb}{0,0.5,0}

\usepackage{listings}

% Python style for highlighting
\newcommand\pythonstyle{\lstset{
language=Python,
basicstyle=\ttm,
morekeywords={self},              % Add keywords here
keywordstyle=\ttb\color{deepblue},
emph={MyClass,__init__},          % Custom highlighting
emphstyle=\ttb\color{deepred},    % Custom highlighting style
stringstyle=\color{deepgreen},
frame=tb,                         % Any extra options here
showstringspaces=false
}}


% Python environment
\lstnewenvironment{python}[1][]
{
\pythonstyle
\lstset{#1}
}
{}

\usetikzlibrary{automata,positioning}

%
% Basic Document Settings
%

\topmargin=-0.45in
\evensidemargin=0in
\oddsidemargin=0in
\textwidth=6.5in
\textheight=9.0in
\headsep=0.25in

\linespread{1.1}

\pagestyle{fancy}
\lhead{\hmwkAuthorName}
\chead{\hmwkClass\ (\hmwkClassInstructor\ \hmwkClassTime): \hmwkTitle}
\rhead{\firstxmark}
\lfoot{\lastxmark}
\cfoot{\thepage}

\renewcommand\headrulewidth{0.4pt}
\renewcommand\footrulewidth{0.4pt}

\setlength\parindent{0pt}

%
% Create Problem Sections
%

\newcommand{\enterProblemHeader}[1]{
    \nobreak\extramarks{}{Problem \arabic{#1} continued on next page\ldots}\nobreak{}
    \nobreak\extramarks{Problem \arabic{#1} (continued)}{Problem \arabic{#1} continued on next page\ldots}\nobreak{}
}

\newcommand{\exitProblemHeader}[1]{
    \nobreak\extramarks{Problem \arabic{#1} (continued)}{Problem \arabic{#1} continued on next page\ldots}\nobreak{}
    \stepcounter{#1}
    \nobreak\extramarks{Problem \arabic{#1}}{}\nobreak{}
}

\setcounter{secnumdepth}{0}
\newcounter{partCounter}
\newcounter{homeworkProblemCounter}
\setcounter{homeworkProblemCounter}{1}
\nobreak\extramarks{Problem \arabic{homeworkProblemCounter}}{}\nobreak{}

%
% Homework Problem Environment
%
% This environment takes an optional argument. When given, it will adjust the
% problem counter. This is useful for when the problems given for your
% assignment aren't sequential. See the last 3 problems of this template for an
% example.
%
\newenvironment{homeworkProblem}[1][-1]{
    \ifnum#1>0
        \setcounter{homeworkProblemCounter}{#1}
    \fi
    \section{Problem \arabic{homeworkProblemCounter}}
    \setcounter{partCounter}{1}
    \enterProblemHeader{homeworkProblemCounter}
}{
    \exitProblemHeader{homeworkProblemCounter}
}

%
% Homework Details
%   - Title
%   - Due date
%   - Class
%   - Section/Time
%   - Instructor
%   - Author
%

\newcommand{\hmwkTitle}{Final Exam}
\newcommand{\hmwkDueDate}{\today}
\newcommand{\hmwkClass}{Number theory}
\newcommand{\hmwkClassTime}{Section A}
\newcommand{\hmwkClassInstructor}{Professor J}
\newcommand{\hmwkAuthorName}{\textbf{V} \and \textbf{U}}

%
% Title Page
%

\title{
    \vspace{2in}
    \textmd{\textbf{\hmwkClass:\ \hmwkTitle}}\\
    \normalsize\vspace{0.1in}\small{Due\ on\ \hmwkDueDate\ at 3:10pm}\\
    \vspace{0.1in}\large{\textit{\hmwkClassInstructor\ \hmwkClassTime}}
    \vspace{3in}
}

\author{\hmwkAuthorName}
\date{}

\renewcommand{\part}[1]{\textbf{\large Part \Alph{partCounter}}\stepcounter{partCounter}\\}

%
% Various Helper Commands
%

% Useful for algorithms
\newcommand{\alg}[1]{\textsc{\bfseries \footnotesize #1}}

% For derivatives
\newcommand{\deriv}[1]{\frac{\mathrm{d}}{\mathrm{d}x} (#1)}

% For partial derivatives
\newcommand{\pderiv}[2]{\frac{\partial}{\partial #1} (#2)}

% Integral dx
\newcommand{\dx}{\mathrm{d}x}

% Alias for the Solution section header
\newcommand{\solution}{\textbf{\large Solution}}

% Probability commands: Expectation, Variance, Covariance, Bias
\newcommand{\E}{\mathrm{E}}
\newcommand{\Var}{\mathrm{Var}}
\newcommand{\Cov}{\mathrm{Cov}}
\newcommand{\Bias}{\mathrm{Bias}}
\newcommand{\suminfty}{\sum_{n=1}^{\infty}}
\newcommand{\Psixn}{\Psi\left(\frac{x}{n}\right)}
\newcommand{\Psii}{\sum_{n=1}^{\infty} \Psi\left(\frac{x}{n}\right) - 2\sum_{n=1}^{\infty} \Psi\left(\frac{x}{2n}\right)}
\newcommand{\Psif}[1]{\Psi\left(\frac{x}{#1}\right)}

\begin{document}

\maketitle

\pagebreak

\begin{homeworkProblem}

Prove that $2$ is primative root of $\mod 11$.

\solution

Note that the set of $\{2^x\}$ for $x \in \{1,2,\cdots, 11\}$ is

\begin{tabular}{|*{11}{c|}}
    $2^1$ & $2^2$ & $2^3$ & $2^4$ & $2^5$ & $2^6$ & $2^7$ & $2^8$ & $2^9$ & $2^{10}$  & $2^{11}$\\
    \hline
    2 & 4 & 8 & 5 & 10 & 9 & 7 & 3 & 6 & 1 & 2 \\
    \hline
\end{tabular}

which is exactly $\{1,2,\cdots, 10\}$
Thus by definition of primative root, the $2$ is primative root of mod 11.

\qed 

\end{homeworkProblem}

\begin{homeworkProblem}
Suppose $p$ and $q$ are primes, $p = 4q+1$. Prove that $q$ is not a primitive root (mod $p$).

\solution:

By Law of Quadratic Reciprocity

\begin{equation}
    \left(\frac{p}{q}\right)\left(\frac{q}{p}\right) = (-1)^{\frac{(p-1)(q-1)}{4}} = (-1)^{q(q-1)}=1
\end{equation}

And the fact that 

\begin{equation}
    p = 4q + 1 \equiv 1^2 \mod q \implies \left(\frac{p}{q}\right) = 1
\end{equation}

We have that

\begin{equation}
    \left(\frac{q}{p}\right) = 1,
\end{equation}

which means there exists $x$ such that $q = x^2 \mod p$

Thus by fermat little theorem,
\begin{equation}
    q^{\frac{p-1}{2}} = x^{p-1} \equiv 1 \mod p
\end{equation}

Hence $q$'s order is not $p-1$, and therefore q is \textit{not} primitive root of $p$
 
\end{homeworkProblem}

\begin{homeworkProblem}

Suppose $p$ and $q$ are primes, $p = 2q + 1, p \equiv 2\mod 5$. Prove that $5$ is a primitive root (mod $p$).

\solution

By Law of Quadratic Reciprocity
\begin{equation}
    \left( \frac{p}{5} \right)  \left( \frac{5}{p} \right)  =\left( -1\right)^{p-1}  =1
\end{equation}

Since $p \equiv 2 \neq x^2 \mod 5$ we have 

\begin{equation}
    \left(\frac{p}{5}\right) = -1
\end{equation}

Thus, 

\begin{equation}
    \left(\frac{5}{p}\right) = -1
\end{equation}

Meaning that $5\not\equiv x^2 \mod p$. Therefore, we can't have
\begin{equation}
    5^{(p-1)/2} = 5^q = 1 \mod p
\end{equation}

Also we can't have
\begin{equation}
    5^2 = 1 \mod p
\end{equation}

or
\begin{equation}
    5 = 1 \mod p
\end{equation}

Otherwise, we will have $p = 2, 3$ in which cases, $p=2q+1$ are not satisfied.
Together, by the Fermat Little Theorem, 
\[\mathrm{Ord}_p(5) = 2q=p-1\]

\end{homeworkProblem}

\begin{homeworkProblem}
Suppose
\[m1 > 2, m2 >2, (m_1,m_2)=1\]
Prove that there is no primitive root (mod $m_1m_2$).

\solution

Let $a$ be coprime to $m_1m_2$. By Euler's theorem, 
\[a^{\phi(m_1)} = 1 \mod m_1, a^{\phi(m_2)} = 1 \mod m_2\]

Let \[L := \mathrm{lcm}(\phi(m_1), \phi(m_2))\], then $a^L = 1 \mod m_1$ and $a^L = 1 \mod m_2$.
Hence

\[a^L = 1 \mod m_1m_2\]

By definition of $\phi$, if $m = p_1^{e_1}\cdots p_k^{e_k}> 2$, 

\[\phi(m) = \prod_{k}p_k^{e_k-1}(p_k-1)\]

is apparently even. 

Thus, 

\[L \le \phi(m_1)\phi(m_2)/2 = \phi(m_1m_2)/2 < \phi(m_1m_2)\]

Apparently, $a$ is not a primitive root. Since $a$ is randomly chosen, we come to conclude that $\mod m_1m_2$ has no primitive root.

\end{homeworkProblem}

\pagebreak

\begin{homeworkProblem}
    Let $\Lambda(n)$ be given by
    \[\Lambda(n) = \begin{cases}
        \log p & \text{if } n = p^k \text{ for some prime } p \text{ and integer } k \ge 1\\
        0 & \text{otherwise}
    \end{cases}\]
    where $p$ denotes a prime. It is known that
    \begin{equation}
        \sum_{n\le x} \Lambda(n)\left[\frac{x}{n}\right] = \sum_{n\le x} \log(n)
    \end{equation}
    Let
    \begin{equation}
        \Psi(x) = \sum_{n\le x} \Lambda(n)
    \end{equation}
    (i) Prove that
    \[\sum_{n\le x} \log(n) = x\log(x) - x + O(\log(x))\]
    (ii) Prove that 
    \[\sum_{n\le x} \Psi\left(\frac{x}{n}\right) = \sum_{n\le x} \Lambda(x)\left[\frac{x}{n}\right]\]

    \solution

    By Stirling's approximation, we have
    \begin{align*}
        \sum_{n\le x} \log(n) &= \int_{0}^{x} \log(x)dx + O(\log(x)) \\
                              &= x\log(x) - x + O(\log(x))
    \end{align*}

    For (ii), we have 
    \begin{align*}
        \sum_{n\le x} \Psi\left(\frac{x}{n}\right) &= \sum_{n\le x} \sum_{m\le x/n} \Lambda(m) \\
                                                   &= \sum_{m\le x} \Lambda(m) \sum_{n\le x/m} 1 \\
                                                   &= \sum_{m\le x} \Lambda(m) \left[\frac{x}{m}\right]\\
                                                   &= \sum_{n\le x} \Lambda(x)\left[\frac{x}{n}\right]
    \end{align*}

    \qed
\end{homeworkProblem}

\pagebreak

\begin{homeworkProblem}
    Prove that
    \begin{equation}
        \sum_{n=1}^{\infty} \Psi\left(\frac{x}{n}\right) - 2\sum_{n=1}^{\infty} \Psi\left(\frac{x}{2n}\right) = x\log(2) + O(\log(x)) \text{  if  } x \ge 4
    \end{equation}
    Note that $\Psi(y) = 0$ if $0 < y \le 1$. The left side of $(5)$ is equal to
    \[\sum_{n\le x} \Psi\left(\frac{x}{n}\right) - 2\sum_{n\le x/2} \Psi\left(\frac{x}{2n}\right)\]
    Thus applying $(4)$, we have
    \begin{align*}
        \sum_{n\le x} \Psi\left(\frac{x}{n}\right) - 2\sum_{n\le x/2} \Psi\left(\frac{x}{2n}\right)
        &= x\log(x) - x + O(\log(x)) - 2\left(\frac{x}{2}\log\left(\frac{x}{2}\right) - \frac{x}{2} + O(\log(x))\right) \\
        &= x\log(2) + O(\log(x))\\
    \end{align*}
\end{homeworkProblem}

\begin{homeworkProblem}
    Prove that
    \begin{equation}
        \Psii \le \Psi(x)
    \end{equation}
    and
    \begin{equation}
        \Psii \ge \Psi(x) - \Psi\left(\frac{x}{2}\right)
    \end{equation}
    \solution

    For (6), we have
    
    \begin{align*}
        \Psii &= \Psi(x) + \Psif{2} + \Psif{3} + \cdots -2\left\{\Psif{2} + \Psif{3}\cdots\right\}\\
              &= \left\{\Psi(x) - \Psif{2}\right\} + \left\{\Psif{3} - \Psif{4}\right\} + \cdots\\
    \end{align*}
    Note that $\Psi(y)$ is increasing for $y > 1$, we have
    \[\Psif{k} - \Psif{k+1} \ge 0\]
    Thus, 
    \[\Psii \ge \Psi(x) - \Psif{2}\]
    For (7), we have
    \begin{align*}
        \Psii &= \Psi(x) + \Psif{2} + \Psif{3} + \cdots -2\left\{\Psif{2} + \Psif{4}\cdots\right\}\\
              &= \Psi(x) - \left\{\Psif{2} - \Psif{3}\right\} - \left\{\Psif{4} - \Psif{5}\right\} - \cdots\\
              &\le \Psi(x)
    \end{align*}
\end{homeworkProblem}

\begin{homeworkProblem}
    Conclude that 
    \begin{equation}
        \Psi(x) \ge x\log(2) + O(\log(x))
    \end{equation}
    and
    \begin{equation}
        \Psi(x) - \Psif{2} \le x\log(2) + O(\log(x))
    \end{equation}
    \solution
    By (6) and (5)
    \begin{align*}
        \Psi(x) &\ge \Psii \\ 
                &= x\log(2) + O(\log(x))
    \end{align*}
    By (7) and (5)
    \begin{align*}
        \Psi(x) - \Psif{2} &\le \Psii \\
                           &= x\log(2) + O(\log(x))
    \end{align*}
\end{homeworkProblem}

\end{document}
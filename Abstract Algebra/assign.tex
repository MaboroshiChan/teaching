\documentclass{article}

\usepackage{fancyhdr}
\usepackage{extramarks}
\usepackage{amsmath}
\usepackage{amsthm}
\usepackage{amsfonts}
\usepackage{tikz}
\usepackage[plain]{algorithm}
\usepackage{algpseudocode}
\usepackage{mathtools}

\usepackage[utf8]{inputenc}

% Default fixed font does not support bold face
\DeclareFixedFont{\ttb}{T1}{txtt}{bx}{n}{12} % for bold
\DeclareFixedFont{\ttm}{T1}{txtt}{m}{n}{12}  % for normal

% Custom colors
\usepackage{color}
\definecolor{deepblue}{rgb}{0,0,0.5}
\definecolor{deepred}{rgb}{0.6,0,0}
\definecolor{deepgreen}{rgb}{0,0.5,0}

\usepackage{listings}

% Python style for highlighting
\newcommand\pythonstyle{\lstset{
language=Python,
basicstyle=\ttm,
morekeywords={self},              % Add keywords here
keywordstyle=\ttb\color{deepblue},
emph={MyClass,__init__},          % Custom highlighting
emphstyle=\ttb\color{deepred},    % Custom highlighting style
stringstyle=\color{deepgreen},
frame=tb,                         % Any extra options here
showstringspaces=false
}}


% Python environment
\lstnewenvironment{python}[1][]
{
\pythonstyle
\lstset{#1}
}
{}

\usetikzlibrary{automata,positioning}

%
% Basic Document Settings
%

\topmargin=-0.45in
\evensidemargin=0in
\oddsidemargin=0in
\textwidth=6.5in
\textheight=9.0in
\headsep=0.25in

\linespread{1.1}

\pagestyle{fancy}
\lhead{\hmwkAuthorName}
\chead{\hmwkClass\ (\hmwkClassInstructor\ \hmwkClassTime): \hmwkTitle}
\rhead{\firstxmark}
\lfoot{\lastxmark}
\cfoot{\thepage}

\renewcommand\headrulewidth{0.4pt}
\renewcommand\footrulewidth{0.4pt}

\setlength\parindent{0pt}

%
% Create Problem Sections
%

\newcommand{\enterProblemHeader}[1]{
    \nobreak\extramarks{}{Problem \arabic{#1} continued on next page\ldots}\nobreak{}
    \nobreak\extramarks{Problem \arabic{#1} (continued)}{Problem \arabic{#1} continued on next page\ldots}\nobreak{}
}

\newcommand{\exitProblemHeader}[1]{
    \nobreak\extramarks{Problem \arabic{#1} (continued)}{Problem \arabic{#1} continued on next page\ldots}\nobreak{}
    \stepcounter{#1}
    \nobreak\extramarks{Problem \arabic{#1}}{}\nobreak{}
}

\setcounter{secnumdepth}{0}
\newcounter{partCounter}
\newcounter{homeworkProblemCounter}
\setcounter{homeworkProblemCounter}{1}
\nobreak\extramarks{Problem \arabic{homeworkProblemCounter}}{}\nobreak{}

%
% Homework Problem Environment
%
% This environment takes an optional argument. When given, it will adjust the
% problem counter. This is useful for when the problems given for your
% assignment aren't sequential. See the last 3 problems of this template for an
% example.
%
\newenvironment{homeworkProblem}[1][-1]{
    \ifnum#1>0
        \setcounter{homeworkProblemCounter}{#1}
    \fi
    \section{Section \arabic{homeworkProblemCounter}}
    \setcounter{partCounter}{1}
    \enterProblemHeader{homeworkProblemCounter}
}{
    \exitProblemHeader{homeworkProblemCounter}
}

%
% Homework Details
%   - Title
%   - Due date
%   - Class
%   - Section/Time
%   - Instructor
%   - Author
%

\newcommand{\hmwkTitle}{Assignment 2\ \#2}
\newcommand{\hmwkDueDate}{\today}
\newcommand{\hmwkClass}{Investigating Dynamic System}
\newcommand{\hmwkClassTime}{Section A}
\newcommand{\hmwkClassInstructor}{Professor J}
\newcommand{\hmwkAuthorName}{\textbf{V} \and \textbf{U}}

%
% Title Page
%

\title{
    \vspace{2in}
    \textmd{\textbf{\hmwkClass:\ \hmwkTitle}}\\
    \normalsize\vspace{0.1in}\small{Due\ on\ \hmwkDueDate\ at 3:10pm}\\
    \vspace{0.1in}\large{\textit{\hmwkClassInstructor\ \hmwkClassTime}}
    \vspace{3in}
}

\author{\hmwkAuthorName}
\date{}

\renewcommand{\part}[1]{\textbf{\large Part \Alph{partCounter}}\stepcounter{partCounter}\\}

%
% Various Helper Commands
%

% Useful for algorithms
\newcommand{\alg}[1]{\textsc{\bfseries \footnotesize #1}}

% For derivatives
\newcommand{\deriv}[1]{\frac{\mathrm{d}}{\mathrm{d}x} (#1)}

% For partial derivatives
\newcommand{\pderiv}[2]{\frac{\partial}{\partial #1} (#2)}

% Integral dx
\newcommand{\dx}{\mathrm{d}x}

% Alias for the Solution section header
\newcommand{\solution}{\textbf{\large Solution}}

% Probability commands: Expectation, Variance, Covariance, Bias
\newcommand{\E}{\mathrm{E}}
\newcommand{\Var}{\mathrm{Var}}
\newcommand{\Cov}{\mathrm{Cov}}
\newcommand{\Bias}{\mathrm{Bias}}

\begin{document}

\maketitle

\pagebreak

\section{Problem 1}

Show that any product of $d$ consecutive integers is divisible by $d!$
(Suggestion:  You know something that should make this easy.)

\begin{proof}
    We need to show $d!$ divides $n(n+1)(n+2)\cdots(n+d-1)$ for all $n\in\mathbb{Z}$.
    We know that 
    \begin{equation*}
        \frac{n(n+1)\cdots (n+d-1)}{d!} = \frac{(n+d-1)!}{d!(n-1)!} = \binom{n+d-1}{d}
    \end{equation*}
    is an integer. Therefore, $d!$ divides $n(n+1)(n+2)\cdots(n+d-1)$ for all $n\in\mathbb{Z}$.
\end{proof}

\section{Problem 2}
Let
\begin{equation}
    f(t) = a_0 + a_1t + \cdots + a_nt^n \in \mathbb{R}[t]
\end{equation}
And let $c=1/(a_n)^{1/n}$.
Let
\begin{equation}
    g(t) = f(ct - a_{n-1}/na_n) = b_0 + b_1t + \cdots + b_{n-1}t^{n-1} + b_nt^n
\end{equation}

(a) Compute $b_0$. 

\solution

\begin{align*}
    b_0 &= g(0) = f(-a_{n-1}/na_n)
\end{align*}

(b) Compute $b_{n-1}$

\solution

We need only to compute the coefficients of $t^{n-1}$ in $g(t)$, which is the coefficient in 
\begin{equation}
    a_{n-1}\left(ct-\frac{a_{n-1}}{na_n}\right)^{n-1} + a_n\left(ct-\frac{a_{n-1}}{na_n}\right)^n
\end{equation}

which is
\begin{align*}
    &a_{n-1}c^{n-1} - \binom{n}{n-1}(c)^{n-1}\left(\frac{a_{n-1}}{na_n}\right)
    = a_{n-1}\frac{1}{a_n} \\
    &= a_{n-1}\left(\frac{a_{n-1}}{a_n} - 1\right)c^{n-1}
\end{align*}

(c) Compute $b_n$

\solution

We need only to compute the coefficients of $t^{n}$ in $g(t)$, which is the coefficient in

\begin{equation}
    a_n\left(ct-\frac{a_{n-1}}{na_n}\right)^n
\end{equation}

which is
\begin{equation}
    b_n=a_nc^n = a_n \cdot \frac{1}{a_n} = 1
\end{equation}

\section{Problem 3}

Let $p$ be prime 

(a) Show that: 
\begin{equation*}
    \binom{p}{k} =_p 0 \text{ for } 0 < k < p
\end{equation*}

\begin{proof}
    We know that 
    \begin{equation*}
        \binom{p}{k} = \frac{p!}{k!(p-k)!}
    \end{equation*}
    Since $p$ is prime, $p$ does not divide $k!$ or $(p-k)!$. Therefore, $p$ divides $p!$ but not $\binom{p}{k}$. Hence, $\binom{p}{k} =_p 0$ for $0 < k < p$.
\end{proof}

(b) Let    $ a=1 + bp^h$      with $h > 0$ and $\gcd(p, b)  =  1$.       Show that:   
$a^p   =   1 + cp^{h+1}$     with   $\gcd(p, c)  =  1$,    unless$ p = 2$ and $h = 1$.

\solution

\begin{proof}
    We know that 
    \begin{align*}
        a^p = (1 + bp^h)^p &= \sum_{k=0}^p \binom{p}{k} (bp^h)^k\\ 
        &= 1 + \sum_{k=1}^{p} \binom{p}{k} (bp^h)^k\\
        &= 1 + \sum_{k=1}^{p} p\left(\binom{p}{k} \Big/ p\right)b^kp^{hk-h}\cdot p^h\\
        &= 1 + p^{h+1}\sum_{k=1}^{p} \binom{p}{k} \Big/ p\cdot b^kp^{hk-h}\\
    \end{align*}
\end{proof}
Let \[c = \sum_{k=1}^{p} \binom{p}{k} \Big/ p\cdot b^kp^{hk-h}\]. Let's look at the first term of the sum. We know that

\[\binom{p}{1} \Big/ p\cdot b^kp^{h-h} = b^k\]

which is coprime to $p$ since $\gcd(p, b) = 1$. Therefore, $c$ is coprime to $p$. Hence, $\gcd(p, c) = 1$.

\qed

(c) In the multiplicative group $(\mathbb{Z}_{p^e})^{*}$, $[1+p]$ has order $p^{e-1}$.

\begin{proof}
    Let $c_0 = 1$, we define $c_n$ in the following way:
    \begin{equation*}
        (1 + c_{n-1}p^{n - 1})^{p} = 1 + c_{n}p^{n}
    \end{equation*}
    Thus,
    \begin{equation*}
        (1 + p)^{p^{e-1}} = 1 + c_{e-1}p^{e} =_p 1 
    \end{equation*}
    Conversely, if $h < e - 1$, then
    \begin{equation*}
        (1 + c_{h}p^{h})^{p} = 1 + c_{h+1}p^{h+1} \neq_p 1
    \end{equation*}
    Therefore, $[1+p]$ has order $p^{e-1}$.
\end{proof}

(d) Prove that $(\mathbb{Z}_{p^e})^{*}$ is cyclic.

\begin{proof}
    From (c), we know that $[1+p]$ has order $p^{e-1}$ which is exactly the order of $(\mathbb{Z}_{p^e})^{*}$. Therefore, $(\mathbb{Z}_{p^e})^{*}$ is cyclic.
\end{proof}

(e) For $e > 2$, prove that
\begin{equation*}
    \left(\mathbb{Z}_{2^e}\right)^{*} \cong \left\{\pm1\right\}\times \left<[5]\right>
\end{equation*}

where $[5]$ has order $2^{e-2}$.
\begin{proof}
    From (b), 
    \begin{align*}
        5^{p^{h}} &= (1 + 2^2)^{p^{h}} = 1 + c2^{2 + h}\\
    \end{align*}
    Thus, $5^{p^h} =_p 1 \iff h = e-2$. Therefore, $[5]$ has order $2^{e-2}$.
    Note that for $e > 2$ $5^{p^e} + 5^{p^e} = 2\cdot 5^{p^e}\neq_{p^e} 0$.
    A homomorphism from $\left\{\pm1\right\}\times \left<[5]\right>$ to $\left(\mathbb{Z}_{2^e}\right)^{*}$ can be given by: $(a, b)\mapsto ab$.
    This map is injective since $5^{p^e} + 5^{p^e} = 2\cdot 5^{p^e}\neq_{p^e} 0$. This map is surjective since $\left|\left\{\pm1\right\}\times \left<[5]\right>\right|=\left|\mathbb{Z}_{p^e}\right|=2^{e-1}$.
    Therefore, \[\left(\mathbb{Z}_{2^e}\right)^{*} \cong \left\{\pm1\right\}\times \left<[5]\right>\].

\end{proof}

\section*{Problem 4}
Recall that a ring homomorphism $f: A\to A'$ satisfies
\begin{itemize}
    \item $f(a + b) = f(a) + f(b)$
    \item $f(ab) = f(a)f(b)$
    \item $f(1) = 1$
\end{itemize}
A composition of ring homomorphisms is a ring homomorphism.

(a) Let $p$ be a prime, Let $\mathbb{Z}_p\subseteq A$ be a commutative ring, so that $pa=0$ for all $a\in A$.
Show that $\phi: A\to A, \phi(a)=a^p$ is a ring homomorphism. And hence likewise for $\phi_{e}(a) = a^{p^e}$

\begin{proof}
    We know that 
    \begin{equation*}
        \phi(a + b) = (a + b)^p = \sum_{k=0}^{p} \binom{p}{k} a^kb^{p-k} = a^p + b^p = \phi(a) + \phi(b)
    \end{equation*}
    and
    \begin{equation*}
        \phi(ab) = (ab)^p = a^pb^p = \phi(a)\phi(b)
    \end{equation*}
    and
    \begin{equation*}
        \phi(1) = 1^p = 1
    \end{equation*}
    Therefore, $\phi$ is a ring homomorphism.
    Likewise, by induction, we assume that $\phi_{e-1}$ is a ring homomorphism. Then,
    \begin{equation*}
        \phi_{e}(a + b) = (a + b)^{p^e} = \left(a^{p^{e-1}} + b^{e^{e-1}}\right)^p = a^{p^e} + b^{p^e} = \phi_{e}(a) + \phi_{e}(b)
    \end{equation*}
    And
    \begin{equation*}
        \phi_{e}(ab) = (ab)^{p^e} = a^{p^e}b^{p^e} = \phi_{e}(a)\phi_{e}(b)
    \end{equation*}
    And
    \begin{equation*}
        \phi_{e}(1) = 1^{p^e} = 1
    \end{equation*}
    Therefore, $\phi_{e}$ is a ring homomorphism.
\end{proof}

(b) Show that, if $n  =  p^em$, then:
\begin{equation*}
    \binom{n}{d} =_p 0 \text{ if } p^e \not | d \text{ and } \binom{n}{d} =_p \binom{m}{d'} \text{ if } d=p^ed'
\end{equation*}

\begin{proof}
    Consider the expansion of $(t + 1)^n$ in $\mathbb{Z}_{p^e}[t]$:
    \begin{equation*}
       \sum_{d=0}^{n}\binom{n}{d}t^d = (1 + t)^n = \left((1 + t)^{p^e}\right)^m = (1 + t^{p^e})^m = \sum_{d'=0}^{m}\binom{m}{d'}t^{p^{e}d'}
    \end{equation*}
    By comparing the coefficients of $t^d$ on both sides, if $p^e \not | d$, then $\binom{n}{d} =_p 0$. If $d = p^ed'$, then $\binom{n}{d} =_p \binom{m}{d'}$.
\end{proof}


\section{Problem 5}
Let $f(t)\in \mathbb{C}[t]$. Show that if $f(\mathbb{Q})\subseteq \mathbb{Q}$, then $f(t)\in \mathbb{Q}[t]$.
(Suggestion: If $\deg(f)  =  d$, apply the proof of the Interpolation theorem to: 
$(0, f(0)), (1, f(1)), \cdots , (d, f(d)))$.

\begin{proof}
    By the Largrange Interpolation Theorem, we can express $f(t)$ as
    \begin{equation*}
        f(t) = \sum_{i=0}^{d}\left( f(i) \prod_{j=0, j\neq i}^{d} \frac{t - j}{i - j}\right)
    \end{equation*}

    Since $f(\mathbb{Q})\subseteq \mathbb{Q}$, we know that $f(i)\in\mathbb{Q}$ for all $i\in\mathbb{Z}$. 
    The operations on the right hand side of the equation generate rational coefficients for $f(t)$.
    Therefore, $f(t)\in\mathbb{Q}[t]$.
\end{proof}

\end{document}
%=======================02-713 LaTeX template, following the 15-210 template==================
%
% You don't need to use LaTeX or this template, but you must turn your homework in as
% a typeset PDF somehow.
%
% How to use:
%    1. Update your information in section "A" below
%    2. Write your answers in section "B" below. Precede answers for all 
%       parts of a question with the command "\question{n}{desc}" where n is
%       the question number and "desc" is a short, one-line description of 
%       the problem. There is no need to restate the problem.
%    3. If a question has multiple parts, precede the answer to part x with the
%       command "\part{x}".
%    4. If a problem asks you to design an algorithm, use the commands
%       \algorithm, \correctness, \runtime to precede your discussion of the 
%       description of the algorithm, its correctness, and its running time, respectively.
%    5. You can include graphics by using the command \includegraphics{FILENAME}
%
\documentclass[11pt]{article}
\usepackage{amsmath,amssymb,amsthm}
\usepackage{graphicx}
\usepackage[margin=1in]{geometry}
\usepackage{fancyhdr}
\usepackage{cancel}
\setlength{\parindent}{0pt}
\setlength{\parskip}{5pt plus 1pt}
\setlength{\headheight}{13.6pt}
\newcommand\question[2]{\vspace{.25in}\hrule\textbf{#1: #2}\vspace{.5em}\hrule\vspace{.10in}}
\renewcommand\part[1]{\vspace{.10in}\textbf{(#1)}}
\newcommand\algorithm{\vspace{.10in}\textbf{Algorithm: }}
\newcommand\correctness{\vspace{.10in}\textbf{Correctness: }}
\newcommand\runtime{\vspace{.10in}\textbf{Running time: }}
\pagestyle{fancyplain}
\lhead{\textbf{\NAME\ (\ANDREWID)}}
\chead{\textbf{HW\HWNUM}}
\rhead{02-713, \today}
\begin{document}\raggedright
%Section A==============Change the values below to match your information==================
\newcommand\NAME{Carl Kingsford}  % your name
\newcommand\ANDREWID{ckingsf}     % your andrew id
\newcommand\HWNUM{1}              % the homework number
%Section B==============Put your answers to the questions below here=======================

% no need to restate the problem --- the graders know which problem is which,
% but replacing "The First Problem" with a short phrase will help you remember
% which problem this is when you read over your homeworks to study.
\newcommand{\sumn}{\sum_{n=0}^{\infty}}
\newcommand{\ea}{e^{\alpha}}
\newcommand{\nea}{e^{-\alpha}}

\question{Problem 4}{}
(a) 
\begin{align*}
    \frac{1}{Z}\frac{\partial Z}{\partial \beta} 
    &= \frac{1}{Z}\frac{\partial \sum_{i}e^{-\beta E_i}}{\partial \beta}\\
    &= \frac{1}{Z}\sum_{i}\frac{\partial e^{-\beta E_i}}{\partial \beta}\\
    &= \frac{1}{Z}\sum_{i}-E_ie^{-\beta E_i} \\
    &= -\left(E_i\sum_{i}\frac{1}{Z}e^{-\beta E_i}\right)\\
    &= -\langle E\rangle
\end{align*}
(b)
\begin{align*}
    S &= -k\sum_{i}p_i\log(p_i)\\
      &= -k\sum_{i}p_i\log\left(\frac{1}{Z}e^{-\beta E_i}\right)\\
      &= -k\sum_{i}p_i\log\left(e^{-\beta E_i}\right)-k\sum_{i} p_i\log(Z)\\
      &= -k\sum_{i}-\beta p_i E_i - k\left(\sum_{i} p_i\right)\log(Z)\\
      &= k\beta \langle E\rangle - k\log(Z)
\end{align*}

\question{problem 3 b}{}
\begin{align*}
    \mathrm{Var}(X) &= E(X^2) - E^2(X)\\
                    &= \sumn n^2p(n)-M^2\\
                    &= \sumn n^2(1-e^{-\alpha})e^{-\alpha n} - M^2\\
                    &= \frac{d^2}{d\alpha^2}t(\alpha)-M^2\\
                    &= \frac{\ea(\ea + 1)(1-\nea)}{(\ea - 1)^3} - M^2\\
                    &= \frac{\ea(\ea + 1)(1-\nea)}{(\ea - 1)^3} - \frac{e^{2\alpha}(1-\nea)^2}{(\ea-1)^4}\\
                    &= \frac{\ea}{(\ea-1)^2}
\end{align*}
Therefore
\begin{align*}
    \sigma=\sqrt{\mathrm{Var}(X)}&=\frac{e^{\alpha/2}}{e^\alpha - 1}\\
                                 &= \frac{1}{2}\frac{1}{\frac{\ea-1}{2e^{\alpha/2}}}\\
                                 &= \frac{1}{2}\frac{1}{e^{\alpha/2}/2-e^{-\alpha/2}/2}\\
                                 &= \frac{1}{2}\frac{1}{\sinh{\frac{\alpha}{2}}}
\end{align*}

\end{document}

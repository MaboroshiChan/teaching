%=======================02-713 LaTeX template, following the 15-210 template==================
%
% You don't need to use LaTeX or this template, but you must turn your homework in as
% a typeset PDF somehow.
%
% How to use:
%    1. Update your information in section "A" below
%    2. Write your answers in section "B" below. Precede answers for all 
%       parts of a question with the command "\question{n}{desc}" where n is
%       the question number and "desc" is a short, one-line description of 
%       the problem. There is no need to restate the problem.
%    3. If a question has multiple parts, precede the answer to part x with the
%       command "\part{x}".
%    4. If a problem asks you to design an algorithm, use the commands
%       \algorithm, \correctness, \runtime to precede your discussion of the 
%       description of the algorithm, its correctness, and its running time, respectively.
%    5. You can include graphics by using the command \includegraphics{FILENAME}
%
\documentclass[11pt]{article}
\usepackage{amsmath,amssymb,amsthm}
\usepackage{graphicx}
\usepackage[margin=1in]{geometry}
\usepackage{fancyhdr}
\usepackage{cancel}
\setlength{\parindent}{0pt}
\setlength{\parskip}{5pt plus 1pt}
\setlength{\headheight}{13.6pt}
\newcommand\question[2]{\vspace{.25in}\hrule\textbf{#1: #2}\vspace{.5em}\hrule\vspace{.10in}}
\renewcommand\part[1]{\vspace{.10in}\textbf{(#1)}}
\newcommand\algorithm{\vspace{.10in}\textbf{Algorithm: }}
\newcommand\correctness{\vspace{.10in}\textbf{Correctness: }}
\newcommand\runtime{\vspace{.10in}\textbf{Running time: }}
\pagestyle{fancyplain}
\lhead{\textbf{\NAME\ (\ANDREWID)}}
\chead{\textbf{HW\HWNUM}}
\rhead{02-713, \today}
\begin{document}\raggedright
%Section A==============Change the values below to match your information==================
\newcommand\NAME{Shi You}  % your name
\newcommand\ANDREWID{0xfffff}     % your andrew id
\newcommand\HWNUM{2}              % the homework number
%Section B==============Put your answers to the questions below here=======================

% no need to restate the problem --- the graders know which problem is which,
% but replacing "The First Problem" with a short phrase will help you remember
% which problem this is when you read over your homeworks to study.
\newcommand{\sumn}{\sum_{n=0}^{\infty}}
\newcommand{\ea}{e^{\alpha}}
\newcommand{\nea}{e^{-\alpha}}
\newcommand{\expo}{\left(-\beta n\epsilon\right)}
\newcommand{\dif}{\mathrm{d}}
\newcommand{\solution}{\textbf{Solution:}}

\question{Problem 1}{}
A perfect crystal comprises $N$ atoms placed on $N$ lattice sites. Defects are formed when an 
atom moves from a lattice site onto and interstitial site. There are $M$ interstitial sites available in the crystal.
Each atom moving from the lattice site to an interstitial site requires an energy $\epsilon$.

(a) Suppose $n$ atoms move onto interstitial sites, requiring a total energy $E=n\epsilon$.
How many ways are there of removing $n$ atoms from the lattice sites? How many ways are there of distributing the $n$ atoms among the $M$ interstitial sites?
By multiplying these together, obtain the total number of ways $W$ of moving $n$ atoms onto interstitial sites.

\solution

The number of ways of removing $n$ atoms from the lattice sites is $N\choose n$.
The number of ways of distributing the $n$ atoms among the $M$ interstitial sites is identical to the number of ways of distributing $n$ identical balls into $M$ distinct boxes, which is $M+n-1\choose n$.
By multiplying these together, we obtain the total number of ways $W$ of moving $n$ atoms onto interstitial sites:
\begin{equation}
    W={N\choose n}{M\choose n}
\end{equation}

(b) Using the microcanonical ensemble, calculate the entropy for $n$ defect atoms, with energy $E=n\epsilon$ and so calculate the temperature.

\solution

The entropy is given by the Boltzmann formula: 
\begin{equation}
    S=k_B\ln W=k_B\ln\left[{N\choose n}{M\choose n}\right]
\end{equation}

By using Stirling's approximation ($\log n! \approx n\log n - n$), we have:
\begin{align*}
    S &= k_B\log W\\
      &= k_B\log\left[{N\choose n}{M\choose n}\right]\\
      &= k_B\log\left[\frac{N!}{n!(N-n)!}\frac{M!}{n!(M-n)!}\right]\\
      &= k_B\log\left[\frac{N!}{n!(N-n)!}\right]+k_B\log\left[\frac{(M+n-1)!}{n!(M-1)!}\right]\\
      &= k_B\left[\log N!-\log n!-\log(N-n)!\right]+k_B\left[\log(M+n-1)!-\log n!-\log(M-1)!\right]\\
      &= k_B\left[N\log N-n\log n-(N-n)\log(N-n)\right]+k_B\left[(M+n-1)\log(M+n-1)-n\log n-(M-1)\log(M-1)\right]\\
      &= k_B\left[2\log n - \log(N-n) - \log(M-n)\right]
\end{align*}

The temperature is given by the inverse of the derivative of the entropy with respect to the energy:
\begin{align*}
    \frac{1}{T}&=\frac{\partial S}{\partial E}=\frac{\partial S}{\partial n}\frac{\partial n}{\partial E}=\frac{k_B}{\epsilon}\frac{\partial }{\partial n}\ln\left[{N\choose n}{M\choose n}\right]\\
               &=\frac{k_B}{\epsilon}\frac{\partial}{\partial n}\left[N\log N-n\log n-(N-n)\log(N-n)+(M+n-1)\log(M+n-1)-n\log n-(M-1)\log(M-1)\right]\\
               &=\frac{k_B}{\epsilon}\left[-(\log n + 1) + -(\log (N-n) + 1) + (\log(M+n-1)+1)-(\log n + 1) - (\log (M-1) + 1)\right]\\
               &=\frac{k_B}{\epsilon} [\log(N-n) + \log(M-n) - 2\log n]
\end{align*}

Thus, 
\begin{align*}
    -\frac{\epsilon}{k_BT}
        &=2\log n - \log(N-n) - \log(M-n)\\
        &=\log\left(\frac{n^2}{(N-n)(M-n)}\right)
\end{align*}

Taking the exponential of both sides, we have:
\begin{align*}
    \exp\left(-\frac{\epsilon}{k_BT}\right) = \frac{n^2}{(N-n)(M-n)}
\end{align*}

\question{Problem 4}{}
(a) Can you show taht the work done by an external pressure $p$ in changing the volue of a
a system by $\Delta V$ is
\begin{equation}
    \Delta W = -p \Delta V.
\end{equation}

\solution

The work done by an external pressure $p$ in changing the volume of a system by $\Delta V$ is given by:
\begin{equation}
    \Delta W = \int_{V_1}^{V_2}p\mathrm{d}V
\end{equation}

Since $p$ is constant, we have:
\begin{align*}
    \Delta W &= \int_{V_1}^{V_2}p\mathrm{d}V\\
             &= p\int_{V_1}^{V_2}\mathrm{d}V\\
             &= p(V_2-V_1)\\
             &= -p(V_1-V_2)\\
             &= -p\Delta V
\end{align*}

(b) What is the conjugate force for an angle of rotation $\theta$ of some object in the system?

\solution

The conjugate force for an angle of rotation $\theta$ of some object in the system is the torque $\tau$. The work done by the torque is given by:

\begin{equation}
    \Delta W = \int_{\theta_1}^{\theta_2}\tau\mathrm{d}\theta
\end{equation}

\question{Problem 6}{}
(a) A single quantum harmonic oscillator has states with energies $E = m\epsilon$ where $m=1,2,\cdots$. Write down the partition function, and energy, the free energy and the entropy of the oscillator.
hence obtain the average energy, the free energy and the entropy of the oscillator.

\solution

The partition function is given by:
\begin{align*}
    Z &= \sum_{m=1}^\infty e^{-\beta m\epsilon}\\
      &= \frac{e^{-\beta\epsilon}}{1-e^{-\beta\epsilon}}
\end{align*}

The energy is given by:
\begin{align*}
    E &= -\frac{\partial}{\partial\beta}\ln Z\\
      &= -\frac{\partial}{\partial\beta}\ln\left(\frac{e^{-\beta\epsilon}}{1-e^{-\beta\epsilon}}\right)\\
      &= -\frac{\partial}{\partial\beta}\left(-\beta\epsilon-\ln(1-e^{-\beta\epsilon})\right)\\
      &= \epsilon+\frac{\epsilon e^{-\beta\epsilon}}{1-e^{-\beta\epsilon}}\\
      &= \epsilon\frac{1+e^{-\beta\epsilon}}{1-e^{-\beta\epsilon}}
\end{align*}

The free energy is given by:
\begin{align*}
    F &= -k_BT\ln Z\\
      &= -k_BT\ln\left(\frac{e^{-\beta\epsilon}}{1-e^{-\beta\epsilon}}\right)\\
      &= -k_BT\left(-\beta\epsilon-\ln(1-e^{-\beta\epsilon})\right)\\
      &= k_BT\left(\beta\epsilon+\ln(1-e^{-\beta\epsilon})\right)\\
      &= k_BT\left(\frac{\epsilon}{k_BT}+\ln(1-e^{-\beta\epsilon})\right)\\
      &= \epsilon+\frac{k_BT}{\beta}\ln(1-e^{-\beta\epsilon})
\end{align*}

The entropy is given by:
\begin{align*}
    S &= \frac{E-F}{T}\\
      &= \frac{\epsilon\frac{1+e^{-\beta\epsilon}}{1-e^{-\beta\epsilon}}-\epsilon-\frac{k_BT}{\beta}\ln(1-e^{-\beta\epsilon})}{T}\\
      &= \frac{\epsilon\frac{1+e^{-\beta\epsilon}}{1-e^{-\beta\epsilon}}-\epsilon-\frac{k_BT}{\beta}\ln(1-e^{-\beta\epsilon})}{k_BT}\\
      &= \frac{\epsilon\frac{1+e^{-\beta\epsilon}}{1-e^{-\beta\epsilon}}-\epsilon-\frac{\epsilon}{\beta}\ln(1-e^{-\beta\epsilon})}{\epsilon}\\
      &= \frac{1+e^{-\beta\epsilon}}{1-e^{-\beta\epsilon}}-1-\frac{1}{\beta}\ln(1-e^{-\beta\epsilon})
\end{align*}

(b) For N identical but distinguishable harmonic oscillators, write down the partition function, and hence obtain the average energy, the free energy and the entropy.

\solution

The partition function is given by:
\begin{align*}
    Z &= \sum_{m=1}^\infty e^{-\beta m\epsilon}\\
      &= \sum_{m=1}^\infty\left(e^{-\beta\epsilon}\right)^m\\
      &= \frac{e^{-\beta\epsilon}}{1-e^{-\beta\epsilon}}
\end{align*}

\question{Problem 10}{}
“A fair dice always has a greater entropy than a biased dice”. Is this statement true? Explain your answer.

\solution

Yes, this statement is true. The entropy of a dice is given by:
\begin{equation}
    S=-k_B\sum_{i=1}^6p_i\ln p_i
\end{equation}

Note that $\ln(x)$ is a concave function (The second derivative of $\ln(x)$ is $-\frac{1}{x^2}$, which is always negative.), using Jensen's inequality, we have:
\begin{align*}
    S &= -k_B\sum_{i=1}^6p_i\ln p_i\\
      &= k_B\sum_{i=1}^6p_i\ln\left(\frac{1}{p_i}\right)\\
      &\leq k_B\ln\left(\sum_{i=1}^6p_i\frac{1}{p_i}\right)\\
      &= k_B\ln 6
\end{align*}

Thus, the entropy of a dice is always less than or equal to $k_B\ln 6$. Since a fair dice has $p_i=\frac{1}{6}$ for all $i$, the entropy of a fair dice is $k_B\ln 6$. Since the entropy of a biased dice is less than or equal to $k_B\ln 6$, the entropy of a fair dice is always greater than the entropy of a biased dice.

\end{document}
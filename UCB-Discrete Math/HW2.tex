%=======================02-713 LaTeX template, following the 15-210 template==================
%
% You don't need to use LaTeX or this template, but you must turn your homework in as
% a typeset PDF somehow.
%
% How to use:
%    1. Update your information in section "A" below
%    2. Write your answers in section "B" below. Precede answers for all 
%       parts of a question with the command "\question{n}{desc}" where n is
%       the question number and "desc" is a short, one-line description of 
%       the problem. There is no need to restate the problem.
%    3. If a question has multiple parts, precede the answer to part x with the
%       command "\part{x}".
%    4. If a problem asks you to design an algorithm, use the commands
%       \algorithm, \correctness, \runtime to precede your discussion of the 
%       description of the algorithm, its correctness, and its running time, respectively.
%    5. You can include graphics by using the command \includegraphics{FILENAME}
%
\documentclass[11pt]{article}
\usepackage{amsmath,amssymb,amsthm}
\usepackage{graphicx}
\usepackage[margin=1in]{geometry}
\usepackage{fancyhdr}
\usepackage{cancel}
\setlength{\parindent}{0pt}
\setlength{\parskip}{5pt plus 1pt}
\setlength{\headheight}{13.6pt}
\newcommand\question[2]{\vspace{.25in}\hrule\textbf{#1: #2}\vspace{.5em}\hrule\vspace{.10in}}
\renewcommand\part[1]{\vspace{.10in}\textbf{(#1)}}
\newcommand\algorithm{\vspace{.10in}\textbf{Algorithm: }}
\newcommand\correctness{\vspace{.10in}\textbf{Correctness: }}
\newcommand\runtime{\vspace{.10in}\textbf{Running time: }}
\pagestyle{fancyplain}
\lhead{\textbf{\NAME\ (\ANDREWID)}}
\chead{\textbf{HW\HWNUM}}
\rhead{02-713, \today}
\begin{document}\raggedright
%Section A==============Change the values below to match your information==================
\newcommand\NAME{Carl Kingsford}  % your name
\newcommand\ANDREWID{ckingsf}     % your andrew id
\newcommand\HWNUM{1}              % the homework number
%Section B==============Put your answers to the questions below here=======================

% no need to restate the problem --- the graders know which problem is which,
% but replacing "The First Problem" with a short phrase will help you remember
% which problem this is when you read over your homeworks to study.


\question{1}{Builup-error} 


\question{2}{Graph proof}
(a)
Prove by contradiction(infinite descending):
Define
\begin{align*}
    d_{-1}(v) := \#\{(w, v)\in E: w\in V\}
\end{align*}
to be the number of in-neighbourhoods of city $v$.
We claim that the city with the most number of in-neighbourhoods satisfies the conditon.

Let $c\in V$ be this vertex. Consider
\[I := \{v\in V: (v, c)\in E\}\]
\[O := \{v\in V: (c, v)\in E\}\]

Cities in $I$ can reach $c$ at one road distance. And $d_{-1}(c) = |I|$ is maximized over $V$.
Pick $o\in O$.

\textbf{Claim: there is $i \in I$ such that $(o, i)\in E$}

If this claim is true, then $o\to i\to c$ are two roads to $c$.
We can easily see that $V = I\cup O$. Thus, a city in either $I$ or $O$ can reach $c$ through at most $2$ roads.

Now we prove this claim. Assume the contrary, that this claim is false.
\[\forall o\in O\forall i\in I, (o, i)\not \in E\]
This implies that $(i, o)\in E$ instead. 

The counting of in-neighbourhoods of $o$ shows that
\begin{align*} 
    d_{-1}(o) &= \#\{(c, o)\} + \sum_{(i, o)\in E}1\\
              &=\#\{(c, o)\} + \sum_{i\in I}1 \\ 
              &= 1 + \sum_{i\in I}1 = 1 + |I| > |I|\\
              &= d_{-1}(c)
\end{align*}
contradicts to the maximality of $d_{-1}(c)$. 

\qed

(b)

By induction, if $m = 1$, according to Euler theorem, there is a euler tour from A to B.
Now assume $m \ge 2$.


Let $(A_i, B_i), 1 \le i\le m$ be $m$ pairs of vertices with odd degrees. According to theorem.XXX, there is a walk
from $A_m$ to $B_m$ denoted by $W_m$. Remove $W_m$ along with $(A_m, B_m)$, we obtain a graph with $m - 1$ pairs of vertices with odd degress.

(c)

(sufficiency)
Base case: there is neither odd cycle or even cycle(no cycle).




Suppose we have a cycle $T_1, T_2, T_3, \cdots, T_n, T_1$,
Its length is even, by condition. We have $n = 2m$ Let 
\[L = \{T_1, T_3, \cdots, T_{n-1}\}\]
\[R = \{T_2, T_4, \cdots, T_{n}\}\]

\question{6}{The bipartite graph}
(a)

\begin{align*}
    \sum_{v \in L} d(v) &= \sum_{v\in L}\sum_{(v, w)\in E}1 \\
                        &= \sum_{v\in L}\sum_{w\in R}1 \\
                        &= \sum_{w\in R}\sum_{v\in L}1 \\ 
                        &= \sum_{w\in R}\sum_{(w, v)\in E}1 \\
                        &= \sum_{w\in R}d(w)
\end{align*}

(b)
By definition,
\[s = \frac{1}{|L|}\sum_{v\in L}d(v), t = \frac{1}{|R|}\sum_{w\in R}d(w)\]
Hence,
\begin{align*}
    \frac{s}{t} &= \frac{\frac{1}{|L|}\cancel{\sum_{v\in L}d(v)}}{\frac{1}{|R|}\cancel{\sum_{w\in R}d(w)}} \\
                &= \frac{\frac{1}{|L|}}{\frac{1}{|R|}} =  \frac{|R|}{|L|}
\end{align*}

(c)

($\Longleftarrow$)
Let $G$ be a graph which can be two colored.
Let's say, that $W\subseteq V$ is the set with every vertices colored white, and
$B\subseteq V$ is the set with every vertices colored black.

Then we have \[V = W \cup B, W\cap B = \{\}\]. Since $G$ is two-colored, we can't have
$\{w_1, w_2\} \in E, \{b_1, b_2\}\in E$ where $w_1, w_2\in W$ and $ b_1, b_2\in B$. Thus,
\[E = W \times B\] By the definition, we conclude that $G$ is bipartite.

$(\implies)$

Let $V = L \times R$. Color $L$ with white, and $R$ with black respectively. The proof that show that $G$ is two-colored is trival.

\end{document}

%=======================02-713 LaTeX template, following the 15-210 template==================
%
% You don't need to use LaTeX or this template, but you must turn your homework in as
% a typeset PDF somehow.
%
% How to use:
%    1. Update your information in section "A" below
%    2. Write your answers in section "B" below. Precede answers for all 
%       parts of a question with the command "\question{n}{desc}" where n is
%       the question number and "desc" is a short, one-line description of 
%       the problem. There is no need to restate the problem.
%    3. If a question has multiple parts, precede the answer to part x with the
%       command "\part{x}".
%    4. If a problem asks you to design an algorithm, use the commands
%       \algorithm, \correctness, \runtime to precede your discussion of the 
%       description of the algorithm, its correctness, and its running time, respectively.
%    5. You can include graphics by using the command \includegraphics{FILENAME}
%
\documentclass[11pt]{article}
\usepackage{amsmath,amssymb,amsthm}
\usepackage{graphicx}
\usepackage[margin=1in]{geometry}
\usepackage{fancyhdr}
\usepackage{mathtools}
\DeclarePairedDelimiter{\ceil}{\lceil}{\rceil}
\setlength{\parindent}{0pt}
\setlength{\parskip}{5pt plus 1pt}
\setlength{\headheight}{13.6pt}
\newcommand\question[2]{\vspace{.25in}\hrule\textbf{#1: #2}\vspace{.5em}\hrule\vspace{.10in}}
\renewcommand\part[1]{\vspace{.10in}\textbf{(#1)}}
\newcommand\algorithm{\vspace{.10in}\textbf{Algorithm: }}
\newcommand\correctness{\vspace{.10in}\textbf{Correctness: }}
\newcommand\runtime{\vspace{.10in}\textbf{Running time: }}
\newcommand{\sumn}{\sum_{k=1}^{n}}
\pagestyle{fancyplain}
\lhead{\textbf{\NAME\ (\ANDREWID)}}
\chead{\textbf{HW\HWNUM}}
\rhead{02-713, \today}
\begin{document}\raggedright
%Section A==============Change the values below to match your information==================
\newcommand\NAME{Carl Kingsford}  % your name
\newcommand\ANDREWID{ckingsf}     % your andrew id
\newcommand\HWNUM{1}              % the homework number
\newcommand\suminf{\sum_{n=0}^{\infty}}
%Section B==============Put your answers to the questions below here=======================

% no need to restate the problem --- the graders know which problem is which,
% but replacing "The First Problem" with a short phrase will help you remember
% which problem this is when you read over your homeworks to study.

\question{1}{The First Problem}
\begin{align*}
    1-C(x) + xC(x)^2 &= 1 - \sum_{n=0}^{\infty}C_nx^{n} + x\left(\sum_{n=0}^{\infty}C_nx^{n}\right)^2\\
                     &= 1 - \suminf C_nx^{n} + \suminf \left(\sum_{k=0}^{n} C_{n-k}C_k\right) x^{n + 1} \\
                     &= 1 - \suminf C_nx^{n} + \sum_{n=1}^{\infty} \left(\sum_{k=0}^{n} C_{n-k}C_k\right) x^{n} \\
                     &= 1 - 1 - \sum_{n=1}^{\infty} C_nx^{n} + \sum_{n=1}^{\infty} \left(\sum_{k=0}^{n} C_{n-k - 1}C_k\right) x^{n} \\
                     &= \sum_{n=1}^{\infty} \left( \left(\sum_{k=0}^{n} C_{n-k}C_k\right) - C_{n}\right)x^{n}\\
                     &= 0
\end{align*}
\question{2}{The second Problem}
Claim: there is a bijection between $\{a_i\}_{i=1}^{n}$ and bracket sequence.
Prove 
\[ 1 \mapsto ( \]
\[-1 \mapsto ) \]
\question{3}{}
We prove that the sequence of non-intersection bijectively corresponds to the ballot sequence.


\question{4}{}
There is a bijection from the ballot number to the $2\times n$ matrices.
Let 
\[\]
\question{6}{}
Consider the sets
\[A_1, A_2, \cdot, A_n\]
where \[A_i = \{a_k| a_k = i\}\]

\end{document}


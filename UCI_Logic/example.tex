%=======================02-713 LaTeX template, following the 15-210 template==================
%
% You don't need to use LaTeX or this template, but you must turn your homework in as
% a typeset PDF somehow.
%
% How to use:
%    1. Update your information in section "A" below
%    2. Write your answers in section "B" below. Precede answers for all 
%       parts of a question with the command "\question{n}{desc}" where n is
%       the question number and "desc" is a short, one-line description of 
%       the problem. There is no need to restate the problem.
%    3. If a question has multiple parts, precede the answer to part x with the
%       command "\part{x}".
%    4. If a problem asks you to design an algorithm, use the commands
%       \algorithm, \correctness, \runtime to precede your discussion of the 
%       description of the algorithm, its correctness, and its running time, respectively.
%    5. You can include graphics by using the command \includegraphics{FILENAME}
%
\documentclass[11pt]{article}
\usepackage{amsmath,amssymb,amsthm}
\usepackage{graphicx}
\usepackage[margin=1in]{geometry}
\usepackage{fancyhdr}
\usepackage{mathtools}
\DeclarePairedDelimiter{\ceil}{\lceil}{\rceil}
\setlength{\parindent}{0pt}
\setlength{\parskip}{5pt plus 1pt}
\setlength{\headheight}{13.6pt}
\newcommand\question[2]{\vspace{.25in}\hrule\textbf{#1: #2}\vspace{.5em}\hrule\vspace{.10in}}
\renewcommand\part[1]{\vspace{.10in}\textbf{(#1)}}
\newcommand\algorithm{\vspace{.10in}\textbf{Algorithm: }}
\newcommand\correctness{\vspace{.10in}\textbf{Correctness: }}
\newcommand\runtime{\vspace{.10in}\textbf{Running time: }}
\newcommand{\sumn}{\sum_{k=1}^{n}}
\pagestyle{fancyplain}
\lhead{\textbf{\NAME\ (\ANDREWID)}}
\chead{\textbf{HW\HWNUM}}
\rhead{02-713, \today}
\begin{document}\raggedright
%Section A==============Change the values below to match your information==================
\newcommand\NAME{Carl Kingsford}  % your name
\newcommand\ANDREWID{ckingsf}     % your andrew id
\newcommand\HWNUM{1}              % the homework number
\newcommand\suminf{\sum_{n=0}^{\infty}}
%Section B==============Put your answers to the questions below here=======================

% no need to restate the problem --- the graders know which problem is which,
% but replacing "The First Problem" with a short phrase will help you remember
% which problem this is when you read over your homeworks to study.

\question{1}{Definition of Injection}

Let $f: X\to Y, x_1, x_2\in X$, then $f(x_1) = f(x_2)$ implies $x_1 = x_2$

\question{2}{Definition of Surjection(on-to)}

Let $f: X\to Y$, $\forall y\in Y\exists x\in X f(x) = y$.

\question{3}{Homework Problem}

Prove that if $f$ is bijection then $f_{\mathrm{cube}}$ is bijection as well.

\textit{Proof of injectivity} Let $x_1, x_2\in \mathbb{R}$. We need to show $f_{\mathrm{cube}}(x_1) = f_{\mathrm{cube}}(x_2)\implies x_1 = x_2$.
By definition of $f_{\mathrm{cube}}$, we have \[f_{\mathrm{cube}}(x_1) = f_{\mathrm{cube}}(x_2)\implies (f(x_1))^3 = (f(x_2))^3\].
Applying the fact in the textbook, we have 
\[(f(x_1))^3 = (f(x_2))^3 \implies ((f(x_1))^3)^\frac{1}{3} = ((f(x_2))^3)^{\frac{1}{3}}\implies f(x_1) = f(x_2)\]
(Need to show $f(x_1) = f(x_2) \implies x_1 = x_2$)

Applying the fact that $f$ is bijection (hence injective), we have that
\[f(x_1)=f(x_2) \implies x_1 = x_2\]

\textit{Proof of surjectivity}

We need to show $\forall y\exists x f_{\text{cube}}(x)=y$. 

Let $y\in \mathbb{R}$. (Need to show $\exists f_{\text{cube}}(x)=y \iff f(x)^3 = y\iff f(x) = y^{\frac{1}{3}}$).
 
Since $y\in \mathbb{R}$, $y^{\frac{1}{3}}$ is also $\in \mathbb{R}$.

Since $f$ is bijection (hence on-to), we have $\exists x\in \mathbb{R}$ such that \[f(x) = y^{\frac{1}{3}}\],
then 
\[f_{\text{cube}}(x) = \left(y^{\frac{1}{3}}\right)^{3}=y\]

This proves the surjectivity of $f_{\text{cube}}$.
\end{document}


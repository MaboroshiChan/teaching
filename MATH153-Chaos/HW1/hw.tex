%=======================02-713 LaTeX template, following the 15-210 template==================
%
% You don't need to use LaTeX or this template, but you must turn your homework in as
% a typeset PDF somehow.
%
% How to use:
%    1. Update your information in section "A" below
%    2. Write your answers in section "B" below. Precede answers for all 
%       parts of a question with the command "\question{n}{desc}" where n is
%       the question number and "desc" is a short, one-line description of 
%       the problem. There is no need to restate the problem.
%    3. If a question has multiple parts, precede the answer to part x with the
%       command "\part{x}".
%    4. If a problem asks you to design an algorithm, use the commands
%       \algorithm, \correctness, \runtime to precede your discussion of the 
%       description of the algorithm, its correctness, and its running time, respectively.
%    5. You can include graphics by using the command \includegraphics{FILENAME}
%
\documentclass[11pt]{article}
\usepackage{amsmath,amssymb,amsthm}
\usepackage{graphicx}
\usepackage[margin=1in]{geometry}
\usepackage{fancyhdr}
\usepackage{cancel}
\setlength{\parindent}{0pt}
\setlength{\parskip}{5pt plus 1pt}
\setlength{\headheight}{13.6pt}
\newcommand\question[2]{\vspace{.25in}\hrule\textbf{#1: #2}\vspace{.5em}\hrule\vspace{.10in}}
\renewcommand\part[1]{\vspace{.10in}\textbf{(#1)}}
\newcommand\algorithm{\vspace{.10in}\textbf{Algorithm: }}
\newcommand\correctness{\vspace{.10in}\textbf{Correctness: }}
\newcommand\runtime{\vspace{.10in}\textbf{Running time: }}
\pagestyle{fancyplain}
\lhead{\textbf{\NAME\ (\ANDREWID)}}
\chead{\textbf{HW\HWNUM}}
\rhead{02-713, \today}
\begin{document}\raggedright
%Section A==============Change the values below to match your information==================
\newcommand\NAME{Carl Kingsford}  % your name
\newcommand\ANDREWID{ckingsf}     % your andrew id
\newcommand\HWNUM{1}              % the homework number
%Section B==============Put your answers to the questions below here=======================

% no need to restate the problem --- the graders know which problem is which,
% but replacing "The First Problem" with a short phrase will help you remember
% which problem this is when you read over your homeworks to study.


\question{1}{}
Let $A\in \mathbb{R}^{2\times 2}$. Let $z(t)$ be a complex-valued function of $t\in \mathbb{R}$.
Let $z(t) = x(t) + iy(t)$, where $x$ and $y$ are real-valued functions, so $x$ is the real part of $z$, and $y$ the
complex part of $z$. Show that
\[\frac{dz}{dt} = Az \text{ if and only if } \frac{dx}{dt} = Ax \text{ and } \frac{dy}{dt} = Ay\]

\textbf{Solution:}

Substitute $z(t) = x(t) + iy(t)$ into $\frac{dz}{dt} = Az$, we have:
\begin{align*}
\frac{dz}{dt} &= \frac{d}{dt}A(x(t) + iy(t))\\
&= A\left(\frac{dx}{dt} + i\frac{dy}{dt}\right)\\
\end{align*}

Also, we have:
\begin{align*}
Az &= A(x(t) + iy(t))\\
&= Ax(t) + iAy(t)\\
\end{align*}

By comparing the real and imaginary parts of $\frac{dz}{dt}$ and $Az$, we have:
\begin{align*}
    \frac{dz}{dt} &= Az\\
    \iff \frac{dx}{dt} + i\frac{dy}{dt} &= Ax(t) + iAy(t)\\
    \iff \frac{dx}{dt} &= Ax(t) \text{ and } \frac{dy}{dt} = Ay(t)\\
\end{align*}
\qed 

\question{2}{}
Let $A\in \mathbb{R}^{2 \times 2}$. Prove that $x$ is a real-valued solution of $\frac{dx}{dt} = Ax$ if and only if
there is a complex-valued solution $z$, with $x(t) = \mathrm{Re}(z)$.

\textbf{Solution:}

\[z(t) = x(t) + ib(t)\],

\[Ax + ibA=A(x(t) + ib(t))=\frac{d(x(t) + ib(t))}{dt} = \frac{dx}{dt}+i\frac{db}{dt}\]
\[Ax = \frac{dx}{dt}\]

\question{3}{}
Now the solution looks like

\[\begin{bmatrix}z_{1}\left( t\right)  \\ z_{2}\left( t\right)  \end{bmatrix} =c_{1}e^{(1+i)t}\begin{bmatrix}i\\ 1\end{bmatrix} +c_{2}e^{\left( 1-i\right)  t}\begin{bmatrix}-i\\ 1\end{bmatrix} \]

Determine the values of $c_1$ and $c_2$ such that the solution is real-valued

\begin{align*}
    \begin{bmatrix}z_{1}\left( t\right)  \\ z_{2}\left( t\right)  \end{bmatrix} 
     &=c_{1}e^{(1+i)t}\begin{bmatrix}i\\ 1\end{bmatrix} +c_{2}e^{\left( 1-i\right)  t}\begin{bmatrix}-i\\ 1\end{bmatrix}\\
     &=e^t\left(c_1(\cos t + i\sin t)\begin{bmatrix}i \\ 1\end{bmatrix} + c_2(\cos(-t) + i\sin(-t))\begin{bmatrix}
       -i \\ 1 
     \end{bmatrix}\right)\\
     &= e^t\left(\begin{bmatrix} ic_1\cos t + i^2c_1\sin t) \\ c_1\cos t + c_1i\sin t\end{bmatrix} + \begin{bmatrix} -ic_2\cos t + c_2i^2\sin t \\ c_2\cos t - ic_2\sin t\end{bmatrix}\right)\\
     &= e^t\left(\begin{bmatrix} -c_1\sin t -c_2\sin t \\ c_1\cos t + c_2\cos t \end{bmatrix} + i \begin{bmatrix} (c_1-c_2) \cos t  \\ (c_1-c_2)\sin t \end{bmatrix}\right)
\end{align*}

\question{4}{}

What are the solutions of $\frac{dx}{dt} = ix$?

\textbf{Solution:}

The solution are 

\[x = ce^{it}=(a+bi)(\cos t +i\sin t)=(a\cos t -b\sin t) + i(a\sin t + b\cos t)\]

(b) Show that a function is the real part of a solution of $dy/dx = ix$ if and only if it it is
linear combination of $\sin t$ and $\cos t$.

\textbf{Solution:}


(c)

Find all real solutions. 

$x = 0$ is the only possible solution by observing the diff-eq.


\end{document}

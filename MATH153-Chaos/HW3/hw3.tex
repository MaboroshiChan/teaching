%=======================02-713 LaTeX template, following the 15-210 template==================
%
% You don't need to use LaTeX or this template, but you must turn your homework in as
% a typeset PDF somehow.
%
% How to use:
%    1. Update your information in section "A" below
%    2. Write your answers in section "B" below. Precede answers for all 
%       parts of a question with the command "\question{n}{desc}" where n is
%       the question number and "desc" is a short, one-line description of 
%       the problem. There is no need to restate the problem.
%    3. If a question has multiple parts, precede the answer to part x with the
%       command "\part{x}".
%    4. If a problem asks you to design an algorithm, use the commands
%       \algorithm, \correctness, \runtime to precede your discussion of the 
%       description of the algorithm, its correctness, and its running time, respectively.
%    5. You can include graphics by using the command \includegraphics{FILENAME}
%
\documentclass[11pt]{article}
\usepackage{amsmath,amssymb,amsthm}
\usepackage{graphicx}
\usepackage[margin=1in]{geometry}
\usepackage{fancyhdr}
\usepackage{cancel}
\setlength{\parindent}{0pt}
\setlength{\parskip}{5pt plus 1pt}
\setlength{\headheight}{13.6pt}
\newcommand\question[2]{\vspace{.25in}\hrule\textbf{#1: #2}\vspace{.5em}\hrule\vspace{.10in}}
\renewcommand\part[1]{\vspace{.10in}\textbf{(#1)}}
\newcommand\algorithm{\vspace{.10in}\textbf{Algorithm: }}
\newcommand\correctness{\vspace{.10in}\textbf{Correctness: }}
\newcommand\runtime{\vspace{.10in}\textbf{Running time: }}
\pagestyle{fancyplain}
\lhead{\textbf{\NAME\ (\ANDREWID)}}
\chead{\textbf{HW\HWNUM}}
\rhead{02-713, \today}
\begin{document}\raggedright
%Section A==============Change the values below to match your information==================
\newcommand\NAME{Carl Kingsford}  % your name
\newcommand\ANDREWID{ckingsf}     % your andrew id
\newcommand\HWNUM{1}              % the homework number
%Section B==============Put your answers to the questions below here=======================

% no need to restate the problem --- the graders know which problem is which,
% but replacing "The First Problem" with a short phrase will help you remember
% which problem this is when you read over your homeworks to study.


\question{3}{}
Solve the linearized damped pendulum equation
\begin{equation}
\ddot{\theta} + \frac{b}{m}\dot{\theta} + \frac{g}{L}\theta = 0
\end{equation}
By transforming into quadratic form, We have to solve 
\begin{equation}
    D^2 + \frac{b}{m}D + \frac{g}{L} = 0
\end{equation}
where $D = \frac{d}{dt}$. The solutions are
\begin{equation}
    D = \frac{-\frac{b}{m} \pm \sqrt{\frac{b^2}{m^2} - 4\frac{g}{L}}}{2}
\end{equation}

Therefore the solutions for the original equation are
\begin{equation}
    \theta = e^{\frac{-b}{2m}t} \left( A_1 e^{\sqrt{\frac{b^2}{4m^2} - \frac{g}{L}}t} + A_2 e^{-\sqrt{\frac{b^2}{4m^2} - \frac{g}{L}}t} \right)
\end{equation}

Note that 

\begin{align*}
    \frac{-b}{m} + \sqrt{\frac{b^2}{m^2} - 4\frac{g}{L}} &< \frac{-b}{m} + \sqrt{\frac{b^2}{m^2}} \\
    &= \frac{-b}{m} + \frac{b}{m} \\
    &= 0
\end{align*}
If $b$ is great enough, the solution will be a decaying exponential.

Physically saying, the pendulum will stop swinging after a while.

\question{4}{}

\end{document}

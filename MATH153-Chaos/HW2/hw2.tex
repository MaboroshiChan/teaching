%=======================02-713 LaTeX template, following the 15-210 template==================
%
% You don't need to use LaTeX or this template, but you must turn your homework in as
% a typeset PDF somehow.
%
% How to use:
%    1. Update your information in section "A" below
%    2. Write your answers in section "B" below. Precede answers for all 
%       parts of a question with the command "\question{n}{desc}" where n is
%       the question number and "desc" is a short, one-line description of 
%       the problem. There is no need to restate the problem.
%    3. If a question has multiple parts, precede the answer to part x with the
%       command "\part{x}".
%    4. If a problem asks you to design an algorithm, use the commands
%       \algorithm, \correctness, \runtime to precede your discussion of the 
%       description of the algorithm, its correctness, and its running time, respectively.
%    5. You can include graphics by using the command \includegraphics{FILENAME}
%
\documentclass[11pt]{article}
\usepackage{amsmath,amssymb,amsthm}
\usepackage{graphicx}
\usepackage[margin=1in]{geometry}
\usepackage{fancyhdr}
\usepackage{cancel}
\setlength{\parindent}{0pt}
\setlength{\parskip}{5pt plus 1pt}
\setlength{\headheight}{13.6pt}
\newcommand\question[2]{\vspace{.25in}\hrule\textbf{#1: #2}\vspace{.5em}\hrule\vspace{.10in}}
\renewcommand\part[1]{\vspace{.10in}\textbf{(#1)}}
\newcommand\algorithm{\vspace{.10in}\textbf{Algorithm: }}
\newcommand\correctness{\vspace{.10in}\textbf{Correctness: }}
\newcommand\runtime{\vspace{.10in}\textbf{Running time: }}
\pagestyle{fancyplain}
\lhead{\textbf{\NAME\ (\ANDREWID)}}
\chead{\textbf{HW\HWNUM}}
\rhead{02-713, \today}
\begin{document}\raggedright
%Section A==============Change the values below to match your information==================
\newcommand\NAME{Carl Kingsford}  % your name
\newcommand\ANDREWID{ckingsf}     % your andrew id
\newcommand\HWNUM{1}              % the homework number
%Section B==============Put your answers to the questions below here=======================

% no need to restate the problem --- the graders know which problem is which,
% but replacing "The First Problem" with a short phrase will help you remember
% which problem this is when you read over your homeworks to study.


\question{1}{The Lotka-Volterra predator-prey model revisited.}
The equation is as presented:
\begin{align*}
  \frac{dx}{dt} &= \gamma(xy - x) \\
  \frac{dy}{dt} &= \frac{1}{\gamma}(y-xy)
\end{align*}

where

\[
   E(x, y) = \frac{1}{\gamma^2} (\ln x - x) + \ln y - y = T(x) + V(y)
\]

We verify that
\begin{align*}
  \frac{dE}{dt} &= \frac{1}{\gamma^2}\left(\frac{\dot{x}}{x} - \dot{x}\right) + \left(\frac{\dot{y}}{y} - \dot{y}\right) \\
                &= \frac{1}{\gamma^2}\left(\gamma(y-1)-\gamma(xy-x)\right) + \left(\frac{1}{\gamma}(1-x)-\frac{1}{\gamma}(y-xy)\right)\\
                &= \frac{1}{\gamma}(y-1+x-xy + 1 - x+ xy -y)\\
                &= 0
\end{align*}

We make the substitution $p=\ln(x), q=\ln(y)$. Then
\[H(p,q) = E(x,y) = \frac{1}{\gamma^2}(p-e^p) + (q - e^q)\]

We compute the Hessian Matrix.

\begin{align*}
  H &= \begin{bmatrix}\frac{\partial^{2} E}{\partial x^{2}} &\frac{\partial^{2} E}{\partial x\partial y} \\ \frac{\partial^{2} E}{\partial x\partial y} &\frac{\partial^{2} E}{\partial y^{2}} \end{bmatrix} \\
    &= \begin{bmatrix}\frac{1}{x^{2}\gamma^{2} } &0\\ 0&\frac{1}{y^{2}} \end{bmatrix}
\end{align*}
Noted that $H$ is symmetric and all its eigenvalues are positive, $H$ is postive definite. 

\question{3}{}

Suppose \[\dot{x} = f(x)\] with a graph. Show that it has no \textit{conserved quantity}.

\textit{proof.}
Assume the contrary, that there is a $E(x)$ such that $\frac{dE}{dt} = 0$.
Then
\[0=\frac{dE}{dt} = \dot{x}\frac{dE}{dx}=f(x)\frac{dE}{dx}\]
From aboved, we can see that whenever $f(x)\neq 0$, $dE/dx = 0$. Since there is a continous interval in which $f(x)\neq 0$, in this area, $E(x)$ is a constant. This creates a contradiction.

\question{4}{Competing species}
Consider the system
\begin{align*}
  \frac{dx}{dt} &= x(1-y) \\
  \frac{dy}{dt} &= y(1-x)
\end{align*}
(a) Find two fixed points,and classify them by computing the Jacobian matrix at each of them and analyzing its eigenvalues.

\textit{solution.}
The fixed points are $(0,0)$ and $(1,1)$. The Jacobian matrix is
\begin{align*}
  J &= \begin{bmatrix}\frac{\partial f}{\partial x} &\frac{\partial f}{\partial y} \\ \frac{\partial g}{\partial x} &\frac{\partial g}{\partial y} \end{bmatrix} \\
    &= \begin{bmatrix}1-y &-x \\ -y &1-x \end{bmatrix}
\end{align*}
At $(0,0)$, $J = \begin{bmatrix}1 &0 \\ 0 &1 \end{bmatrix}$, which has eigenvalues $1$ and $1$. So it is a unstable node.
At $(1,1)$, $J = \begin{bmatrix}0 &-1 \\ -1 &0 \end{bmatrix}$, which has eigenvalues $1$ and $-1$. So it is a saddle point.

(b) Show that there is no conserved quantity for this system.

\textit{solution.}


\end{document}

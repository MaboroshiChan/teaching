%=======================02-713 LaTeX template, following the 15-210 template==================
%
% You don't need to use LaTeX or this template, but you must turn your homework in as
% a typeset PDF somehow.
%
% How to use:
%    1. Update your information in section "A" below
%    2. Write your answers in section "B" below. Precede answers for all 
%       parts of a question with the command "\question{n}{desc}" where n is
%       the question number and "desc" is a short, one-line description of 
%       the problem. There is no need to restate the problem.
%    3. If a question has multiple parts, precede the answer to part x with the
%       command "\part{x}".
%    4. If a problem asks you to design an algorithm, use the commands
%       \algorithm, \correctness, \runtime to precede your discussion of the 
%       description of the algorithm, its correctness, and its running time, respectively.
%    5. You can include graphics by using the command \includegraphics{FILENAME}
%
\documentclass[11pt]{article}
\usepackage{amsmath,amssymb,amsthm}
\usepackage{graphicx}
\usepackage[margin=1in]{geometry}
\usepackage{fancyhdr}
\usepackage{cancel}
\setlength{\parindent}{0pt}
\setlength{\parskip}{5pt plus 1pt}
\setlength{\headheight}{13.6pt}
\newcommand\question[2]{\vspace{.25in}\hrule\textbf{#1: #2}\vspace{.5em}\hrule\vspace{.10in}}
\renewcommand\part[1]{\vspace{.10in}\textbf{(#1)}}
\newcommand\algorithm{\vspace{.10in}\textbf{Algorithm: }}
\newcommand\correctness{\vspace{.10in}\textbf{Correctness: }}
\newcommand\runtime{\vspace{.10in}\textbf{Running time: }}
\pagestyle{fancyplain}
\lhead{\textbf{\NAME\ (\ANDREWID)}}
\chead{\textbf{HW\HWNUM}}
\rhead{02-713, \today}
\begin{document}\raggedright
%Section A==============Change the values below to match your information==================
\newcommand\NAME{Carl Kingsford}  % your name
\newcommand\ANDREWID{ckingsf}     % your andrew id
\newcommand\HWNUM{1}              % the homework number
%Section B==============Put your answers to the questions below here=======================

% no need to restate the problem --- the graders know which problem is which,
% but replacing "The First Problem" with a short phrase will help you remember
% which problem this is when you read over your homeworks to study.
\newcommand{\sumn}{\sum_{n=0}^{\infty}}
\newcommand{\ea}{e^{\alpha}}
\newcommand{\nea}{e^{-\alpha}}

\question{Problem 1}{}
Differentiate the following functions with respect to $x$:
\[y=\frac{8x}{9-\cot x}\]

\textbf{Solution}.
We use the quotient rule:
\begin{align*}
\frac{d}{dx}\left(\frac{f}{g}\right)&=\frac{f'g-fg'}{g^2}\\
\end{align*}

where $f=8x$ and $g=9-\cot x$.  We compute the derivatives of $f$ and $g$:

\begin{align*}
y'&=\frac{(8x)'(9-\cot x)-(8x)(9-\cot x)'}{(9-\cot x)^2}\\
  &= \frac{8(9-\cot x)-8x\cot x'}{(9-\cot x)^2}\\
  &= \frac{8(9-\cot x)+8x\csc^2 x}{(9-\cot x)^2}\\
  &= \frac{72-8\cot x+8x\csc^2 x}{(9-\cot x)^2}\\
\end{align*}

\question{Problem 2}{}
Calculate the area of one petal of the three-petal rose $r=\cos 3\theta$.

\textbf{Solution}.

We use the formula for the area of a polar curve:
\[A=\frac{1}{2}\int_{\alpha}^{\beta}r^2d\theta\]

where $\alpha$ and $\beta$ are the angles at which the curve intersects itself.  In this case, we have $\alpha=\pi/2$ and $\beta=5\pi/6$.
The reason for which $\beta = \frac{5\pi}{6}$ is that the curve intersects itself at $\theta = \frac{\pi}{6}$ and $\theta = \frac{5\pi}{6}$, and the area of one petal is the area between these two points.

Thus, we have:
\begin{align*}
    A &= \frac{1}{2}\int_{\pi/2}^{5\pi/6} \cos^2 3\theta d\theta\\
      &= \frac{1}{2}\int_{\pi/2}^{5\pi/6} \frac{1+\cos 6\theta}{2} d\theta\\
      &= \frac{1}{4}\int_{\pi/2}^{5\pi/6} (1+\cos 6\theta) d\theta\\
      &= \frac{1}{4}\left(\int_{\pi/2}^{5\pi/6} 1 d\theta + \int_{\pi/2}^{5\pi/6} \cos 6\theta d\theta\right)\\
      &= \frac{1}{4}\left(\theta + \frac{\sin 6\theta}{6}\right)_{\pi/2}^{5\pi/6}\\
      &= \frac{1}{4}\left(\frac{5\pi}{6} + \frac{\sin 5\pi}{6} - \frac{\pi}{2} - \frac{\sin \pi}{6}\right)\\
      &= \frac{\pi}{12}\\
\end{align*}

\question{4}{}
Comparison test and ratio test

\question{5}{}
By the power rule for integrals, we have:
\[\int_{0}^{\infty}\frac{1}{x^{3/4}}=\lim_{b\rightarrow\infty}\int_{0}^{b}\frac{1}{x^{3/4}} = \lim_{b\rightarrow\infty}\frac{4}{1/4}b^{1/4} = \infty\]
Thus (a) diverges.
\[\int_{0}^{\infty}\frac{1}{e^{-2x}}= \lim_{b\rightarrow\infty}\int_{0}^{b}\frac{1}{e^{-2x}} = \lim_{b\rightarrow\infty}\frac{1}{-1/2}e^{-2x} = -2\]
Thus (b) converges.

\question{7}{}
Determine whether the following series converge or diverge:
\[\sum_{n=0}^{\infty}\frac{5e^{-2n^2}}{n^2-1}\]
\textbf{Solution}.
We use the limit comparison test.  We compare the series to $\sum_{n=0}^{\infty}\frac{1}{n^2}$, which converges.  We have:
\[\lim_{n\rightarrow\infty}\frac{5e^{-2n^2}/(n^2-1)}{1/n^2} = \lim_{n\rightarrow\infty}\frac{5e^{-2n^2}n^2}{n^2-1} = \lim_{n\rightarrow\infty}\frac{5e^{-2n^2}}{1-1/n^2} = 5\]

\question{3}{}

Find the volume of region created by rotating the region bounded by $y=-x^2+4x+1$ and $y=1$ about the $y=1$-axis.

\textbf{Solution}.
We use the formula for volume of a solid of revolution:
\[V=\int_{a}^{b}\pi(f(x)-g(x))^2dx\]

where $R$ is the outer radius and $r$ is the inner radius.  In this case, we have $R=1$ and $r=1-x^2+4x+1=2-x^2+4x$, and $f(x)=-x^2+4x+1$, $g(x) = 1$ We determine $a$, and $b$:
\begin{align*}
    -x^2+4x+1 &= 1\\
    -x^2+4x &= 0\\
    x^2-4x &= 0\\
    x(x-4) &= 0\\
    x &= 0, 4\\
\end{align*}

Thus we have:
\begin{align*}
  V &= \int_{0}^{4} \pi(f(x)-g(x))^2dx \\
    &= \int_{0}^{4} \pi(x^2+4x)^2dx \\
    &= \pi\int_{0}^{4} (x^4+8x^3+16x^2)dx \\
    &= \pi\left(\frac{x^5}{5}+2x^4+\frac{16x^3}{3}\right)_{0}^{4} \\
    &= \pi\left(\frac{1024}{15}+128+\frac{2048}{3}\right) \\
    &= \frac{15872\pi}{15} \\
\end{align*}

\question{Problem 4}{}

\textbf{Solution}.
We use the formula for hydrostatic force:
\[F=\int_{a}^{b}w(y)ydA\]

where $w(y)$ is the weight density of the fluid at depth $y$ and $dA$ is the area of the plate at depth $y$. 
In this case, we have $w(y)=\gamma y$ and $A=(3-y)^2/3\implies dA=\frac{2(3-y)}{3}dy$.

Thus, we have:
\begin{align*}
    F &= \int_{0}^{3} \gamma y \frac{2(3-y)}{3}dy\\
      &= \frac{2\gamma}{3}\int_{0}^{3} (3y-y^2)dy\\
      &= \frac{2\gamma}{3}\left(\frac{3y^2}{2}-\frac{y^3}{3}\right)_{0}^{3}\\
      &= \frac{2\gamma}{3}\left(\frac{27}{2}-9\right)\\
      &= 3\gamma \\
\end{align*}
Put $\gamma = 90\mathrm{lb}/\mathrm{ft^3}$
We have the force is $F=270\mathrm{lb}$.

\end{document}

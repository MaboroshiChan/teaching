\documentclass{article}

\usepackage{fancyhdr}
\usepackage{extramarks}
\usepackage{amsmath}
\usepackage{amsthm}
\usepackage{amsfonts}
\usepackage{tikz}
\usepackage[plain]{algorithm}
\usepackage{algpseudocode}
\usepackage{mathtools}

\usepackage[utf8]{inputenc}

% Default fixed font does not support bold face
\DeclareFixedFont{\ttb}{T1}{txtt}{bx}{n}{12} % for bold
\DeclareFixedFont{\ttm}{T1}{txtt}{m}{n}{12}  % for normal

% Custom colors
\usepackage{color}
\definecolor{deepblue}{rgb}{0,0,0.5}
\definecolor{deepred}{rgb}{0.6,0,0}
\definecolor{deepgreen}{rgb}{0,0.5,0}

\usepackage{listings}

% Python style for highlighting
\newcommand\pythonstyle{\lstset{
language=Python,
basicstyle=\ttm,
morekeywords={self},              % Add keywords here
keywordstyle=\ttb\color{deepblue},
emph={MyClass,__init__},          % Custom highlighting
emphstyle=\ttb\color{deepred},    % Custom highlighting style
stringstyle=\color{deepgreen},
frame=tb,                         % Any extra options here
showstringspaces=false
}}


% Python environment
\lstnewenvironment{python}[1][]
{
\pythonstyle
\lstset{#1}
}
{}

\usetikzlibrary{automata,positioning}

%
% Basic Document Settings
%

\topmargin=-0.45in
\evensidemargin=0in
\oddsidemargin=0in
\textwidth=6.5in
\textheight=9.0in
\headsep=0.25in

\linespread{1.1}

\pagestyle{fancy}
\lhead{\hmwkAuthorName}
\chead{\hmwkClass\ (\hmwkClassInstructor\ \hmwkClassTime): \hmwkTitle}
\rhead{\firstxmark}
\lfoot{\lastxmark}
\cfoot{\thepage}

\renewcommand\headrulewidth{0.4pt}
\renewcommand\footrulewidth{0.4pt}

\setlength\parindent{0pt}

%
% Create Problem Sections
%

\newcommand{\enterProblemHeader}[1]{
    \nobreak\extramarks{}{Problem \arabic{#1} continued on next page\ldots}\nobreak{}
    \nobreak\extramarks{Problem \arabic{#1} (continued)}{Problem \arabic{#1} continued on next page\ldots}\nobreak{}
}

\newcommand{\exitProblemHeader}[1]{
    \nobreak\extramarks{Problem \arabic{#1} (continued)}{Problem \arabic{#1} continued on next page\ldots}\nobreak{}
    \stepcounter{#1}
    \nobreak\extramarks{Problem \arabic{#1}}{}\nobreak{}
}

\setcounter{secnumdepth}{0}
\newcounter{partCounter}
\newcounter{homeworkProblemCounter}
\setcounter{homeworkProblemCounter}{1}
\nobreak\extramarks{Problem \arabic{homeworkProblemCounter}}{}\nobreak{}

%
% Homework Problem Environment
%
% This environment takes an optional argument. When given, it will adjust the
% problem counter. This is useful for when the problems given for your
% assignment aren't sequential. See the last 3 problems of this template for an
% example.
%
\newenvironment{homeworkProblem}[1][-1]{
    \ifnum#1>0
        \setcounter{homeworkProblemCounter}{#1}
    \fi
    \section{Problem \arabic{homeworkProblemCounter}}
    \setcounter{partCounter}{1}
    \enterProblemHeader{homeworkProblemCounter}
}{
    \exitProblemHeader{homeworkProblemCounter}
}

%
% Homework Details
%   - Title
%   - Due date
%   - Class
%   - Section/Time
%   - Instructor
%   - Author
%

\newcommand{\hmwkTitle}{Final Exam}
\newcommand{\hmwkDueDate}{\today}
\newcommand{\hmwkClass}{Number theory}
\newcommand{\hmwkClassTime}{Section A}
\newcommand{\hmwkClassInstructor}{Professor J}
\newcommand{\hmwkAuthorName}{\textbf{V} \and \textbf{U}}

%
% Title Page
%

\title{
    \vspace{2in}
    \textmd{\textbf{\hmwkClass:\ \hmwkTitle}}\\
    \normalsize\vspace{0.1in}\small{Due\ on\ \hmwkDueDate\ at 3:10pm}\\
    \vspace{0.1in}\large{\textit{\hmwkClassInstructor\ \hmwkClassTime}}
    \vspace{3in}
}

\author{\hmwkAuthorName}
\date{}

\renewcommand{\part}[1]{\textbf{\large Part \Alph{partCounter}}\stepcounter{partCounter}\\}

%
% Various Helper Commands
%

% Useful for algorithms
\newcommand{\alg}[1]{\textsc{\bfseries \footnotesize #1}}

% For derivatives
\newcommand{\deriv}[1]{\frac{\mathrm{d}}{\mathrm{d}x} (#1)}

% For partial derivatives
\newcommand{\pderiv}[2]{\frac{\partial}{\partial #1} (#2)}

% Integral dx
\newcommand{\dx}{\mathrm{d}x}

% Alias for the Solution section header
\newcommand{\solution}{\textbf{\large Solution}}

% Probability commands: Expectation, Variance, Covariance, Bias
\newcommand{\E}{\mathrm{E}}
\newcommand{\Var}{\mathrm{Var}}
\newcommand{\Cov}{\mathrm{Cov}}
\newcommand{\Bias}{\mathrm{Bias}}
\newcommand{\suminfty}{\sum_{n=1}^{\infty}}
\newcommand{\Psixn}{\Psi\left(\frac{x}{n}\right)}
\newcommand{\Psii}{\sum_{n=1}^{\infty} \Psi\left(\frac{x}{n}\right) - 2\sum_{n=1}^{\infty} \Psi\left(\frac{x}{2n}\right)}
\newcommand{\Psif}[1]{\Psi\left(\frac{x}{#1}\right)}

\begin{document}

\maketitle

\pagebreak

\begin{homeworkProblem}
    Calculate $\mathbf{Y}$. Provide all steps you used to get the result.
    \begin{equation}
        (\mathbf{I} - \mathbf{A})\mathbf{Y} = \mathbf{X}^{T}
    \end{equation}
where
    \[
        \mathbf{A} = \begin{bmatrix}
            0 & -1 & -1 \\
            -6 & -4 & -4 \\
            -13 & -10 & 1
        \end{bmatrix}
    \]
$\mathrm{X} = [2,1,3]$, and $\mathbf{I}$ is the identity matrix.

\solution

To calculate $\mathbf{Y}$ in the equation $(\mathbf{I} - \mathbf{A})\mathbf{Y} = \mathbf{X}^T$, we can follow these steps:

1. Define the matrices:
   \[
   \mathbf{A} = \begin{bmatrix}
       0 & -1 & -1 \\
       -6 & -4 & -4 \\
       -13 & -10 & 1
   \end{bmatrix}, \quad \mathbf{X}^T = \begin{bmatrix} 2 \\ 1 \\ 3 \end{bmatrix}, \quad \mathbf{I} = \begin{bmatrix}
       1 & 0 & 0 \\
       0 & 1 & 0 \\
       0 & 0 & 1
   \end{bmatrix}
   \]

2. Calculate the matrix $(\mathbf{I} - \mathbf{A})$ by subtracting $\mathbf{A}$ from $\mathbf{I}$:
   \[
   \mathbf{I} - \mathbf{A} = \begin{bmatrix}
       1 & 0 & 0 \\
       0 & 1 & 0 \\
       0 & 0 & 1
   \end{bmatrix} - \begin{bmatrix}
       0 & -1 & -1 \\
       -6 & -4 & -4 \\
       -13 & -10 & 1
   \end{bmatrix} = \begin{bmatrix}
       1 & 1 & 1 \\
       6 & 5 & 4 \\
       13 & 10 & 0
   \end{bmatrix}
   \]

3. Rewrite the equation as a matrix equation:
   \[
   (\mathbf{I} - \mathbf{A})\mathbf{Y} = \mathbf{X}^T
   \]
   becomes
   \[
   \begin{bmatrix}
       1 & 1 & 1 \\
       6 & 5 & 4 \\
       13 & 10 & 0
   \end{bmatrix}\mathbf{Y} = \begin{bmatrix} 2 \\ 1 \\ 3 \end{bmatrix}
   \]

Let's solve the equation using Gaussian elimination:

   1. Set up the augmented matrix by combining the matrix $(\mathbf{I} - \mathbf{A})$ and the vector $\mathbf{X}^T$:
      \[
      \left[\begin{array}{ccc|c}
      1 & 1 & 1 & 2 \\
      6 & 5 & 4 & 1 \\
      13 & 10 & 0 & 3 \\
      \end{array}\right]
      \]
   
   2. Perform row operations to transform the augmented matrix into row-echelon form:
   
      R2 = R2 - 6R1\\
      R3 = R3 - 13R1
   
      \[
      \left[\begin{array}{ccc|c}
      1 & 1 & 1 & 2 \\
      0 & -1 & -2 & -11 \\
      0 & -3 & -13 & -23 \\
      \end{array}\right]
      \]
   
      R3 = R3 - 3R2
   
      \[
      \left[\begin{array}{ccc|c}
      1 & 1 & 1 & 2 \\
      0 & -1 & -2 & -11 \\
      0 & 0 & -7 & 10\\
      \end{array}\right]
      \]
   
      R3 = (-1/7)R3
   
      \[
      \left[\begin{array}{ccc|c}
      1 & 1 & 1 & 2 \\
      0 & -1 & -2 & -11 \\
      0 & 0 & 1 & -10/7 \\
      \end{array}\right]
      \]
   
      R2 = R2 + 2R3\\
      R1 = R1 - R3
   
      \[
      \left[\begin{array}{ccc|c}
      1 & 1 & 0 & 24/7 \\
      0 & -1 & 0 & -97/7 \\
      0 & 0 & 1 & -10/7 \\
      \end{array}\right]
      \]
   
      R1 = R1 + R2
   
      \[
      \left[\begin{array}{ccc|c}
      1 & 0 & 0 & -73/7 \\
      0 & -1 & 0 & -97/7 \\
      0 & 0 & 1 & -10/7 \\
      \end{array}\right]
      \]
   
   3. The row-echelon form of the augmented matrix gives us the solution to the system of equations. The values in the rightmost column correspond to the elements of $\mathbf{Y}$.
   
      Therefore, the solution is:
      \[
      \mathbf{Y} = \begin{bmatrix}
      -73/7 \\
      97/7 \\
      -10/7 \\
      \end{bmatrix}
      \]
   
   Thus, $\mathbf{Y} = \left[-\frac{73}{7}, \frac{97}{7}, -\frac{10}{7}\right]$.

\end{homeworkProblem}

\begin{homeworkProblem}
    Find the solution of the following system:
    \begin{equation}
        \begin{cases}
            x+2y + 3z = 143\\
            x + 2y + z = 103\\
            x + y + 2z = 11
        \end{cases}
    \end{equation}

\solution

First off, we set up the augmented matrix:
\[
\left[\begin{array}{ccc|c}
1 & 2 & 3 & 143 \\
1 & 2 & 1 & 103 \\
1 & 1 & 2 & 11 \\
\end{array}\right]
\]

Then, we perform row operations to transform the augmented matrix into row-echelon form:

R3 = R1 - R2\\
R2 = R2 - R3\\

\[
\left[\begin{array}{ccc|c}
1 & 2 & 3 & 143 \\
0 & 1 & -1 & 92 \\
0 & 0 & 2 & 40 \\
\end{array}\right]
\]

R3 = (1/2)R3\\

\[
\left[\begin{array}{ccc|c}
1 & 2 & 3 & 143 \\
0 & 1 & -1 & 92 \\
0 & 0 & 1 & 20 \\
\end{array}\right]
\]

R2 = R2 + R3\\
R1 = R1 - 3R3\\

\[
\left[\begin{array}{ccc|c}
1 & 2 & 0 & 83 \\
0 & 1 & 0 & 112 \\
0 & 0 & 1 & 20 \\
\end{array}\right]
\]

R1 = R1 - 2R2\\

\[
\left[\begin{array}{ccc|c}
1 & 0 & 0 & -141 \\
0 & 1 & 0 & 112 \\
0 & 0 & 1 & 20 \\
\end{array}\right]
\]

Therefore, the solution is:
\[
    \begin{cases}
        x = -141\\
        y = 112\\
        z = 20
    \end{cases}
\]

\end{homeworkProblem}

\end{document}
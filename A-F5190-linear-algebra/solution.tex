%=======================02-713 LaTeX template, following the 15-210 template==================
%
% You don't need to use LaTeX or this template, but you must turn your homework in as
% a typeset PDF somehow.
%
% How to use:
%    1. Update your information in section "A" below
%    2. Write your answers in section "B" below. Precede answers for all 
%       parts of a question with the command "\question{n}{desc}" where n is
%       the question number and "desc" is a short, one-line description of 
%       the problem. There is no need to restate the problem.
%    3. If a question has multiple parts, precede the answer to part x with the
%       command "\part{x}".
%    4. If a problem asks you to design an algorithm, use the commands
%       \algorithm, \correctness, \runtime to precede your discussion of the 
%       description of the algorithm, its correctness, and its running time, respectively.
%    5. You can include graphics by using the command \includegraphics{FILENAME}
%
\documentclass[11pt]{article}
\usepackage{amsmath,amssymb,amsthm}
\usepackage{graphicx}
\usepackage[margin=1in]{geometry}
\usepackage{fancyhdr}
\usepackage{cancel}
\setlength{\parindent}{0pt}
\setlength{\parskip}{5pt plus 1pt}
\setlength{\headheight}{13.6pt}
\newcommand\question[2]{\vspace{.25in}\hrule\textbf{#1: #2}\vspace{.5em}\hrule\vspace{.10in}}
\renewcommand\part[1]{\vspace{.10in}\textbf{(#1)}}
\newcommand\algorithm{\vspace{.10in}\textbf{Algorithm: }}
\newcommand\correctness{\vspace{.10in}\textbf{Correctness: }}
\newcommand\runtime{\vspace{.10in}\textbf{Running time: }}
\pagestyle{fancyplain}
\lhead{\textbf{\NAME\ (\ANDREWID)}}
\chead{\textbf{HW\HWNUM}}
\rhead{02-713, \today}
\begin{document}\raggedright
%Section A==============Change the values below to match your information==================
\newcommand\NAME{Carl Kingsford}  % your name
\newcommand\ANDREWID{ckingsf}     % your andrew id
\newcommand\HWNUM{1}              % the homework number
%Section B==============Put your answers to the questions below here=======================

% no need to restate the problem --- the graders know which problem is which,
% but replacing "The First Problem" with a short phrase will help you remember
% which problem this is when you read over your homeworks to study.
\newcommand{\sumn}{\sum_{n=0}^{\infty}}
\newcommand{\ea}{e^{\alpha}}
\newcommand{\nea}{e^{-\alpha}}
\newcommand{\vecv}{\mathbf{v}}
\newcommand{\veca}{\mathbf{a}}
\newcommand{\vecb}{\mathbf{b}}
\newcommand{\rank}{\mathrm{rank}}
\newcommand{\nullity}{\mathrm{nullity}}


\question{Problem 1}{(a) Keywords: linear span, matrix multiplication, addition of vector space}

Let $A$ be an $m \times k$-matrix, and let $B$ be a $k \times n$-matrix.
(a) Prove that the column space of $AB$ is contained in the column space of A.

Subset proof format: Prove the implication 

\[\vec{v} \in Col(AB) \implies \vec{v} \in Col(A)\].

\textbf{Proof.}

Definition of $Col(A)$:
\[Col(A) := \{A\vecv| \vecv \in \mathbb{R}^n\}\]
\[Row(A) := \{A^T\vecv| \vecv \in \mathbb{R}^n\}\]

Let $A = [\veca_1, \veca_2, \cdots, \veca_{k}]$, $B = [\vecb_1, \vecb_2, \cdots, \vecb_{n}]$
Then
\[AB = [A\vecb_1, A\vecb_2, A\vecb_3,\cdots, A\vecb_n]\]

Let $\vec{v} \in Col(AB)$, then
\[\vec{v} = \sum_{i=1}^{n}c_iA\vecb_i\]

By \textbf{conservation of addition of vector space}.

\question{Problem 1}{(b) Keywords: Definition of Nullspace.}
Assume that $k = m$ and that $A$ is invertible. Prove that the null space of $AB$ is equal to the null space of $B$.

\textbf{Proof.($\implies$)}

Let $x\in \mathrm{Null}(B)$, then we can easily see that $ABx=A(Bx)=A\cdot\vec{0}=\vec{0}$
Thus, $x\in \mathrm{Null}(AB)$

($\Longleftarrow$)

Let $x\in \mathrm{Null}(AB)$, then $ABx = 0$.
Since $A$ is invertible, $\mathrm{Null}(A) = \{\mathbf{0}\}$.
This forces $Bx=0$. Therefore, $x\in \mathrm{Null}(B)$.
\question{Problem 1}{(c) keywords: Rank-nullity theorem, Dimension and rank difference}
We know that $B$ and $AB$ is $m\times n$.
Therefore $\dim{AB}=\dim{B} = n$.
From part (b) we know that $\nullity(AB)=\nullity(B)$.
Thus, by rank-nullity theorem,
\begin{align*}
    \rank(AB)&=\dim(AB)-\nullity(AB)\\
             &=\dim(B)-\nullity(B)\\
             &=\rank(B)
\end{align*}

\textit{Comment: Rank of a matrix is the actual dimension of this matrix} 

\question{Problem 2}{2}



\end{document}

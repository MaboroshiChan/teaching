%=======================02-713 LaTeX template, following the 15-210 template==================
%
% You don't need to use LaTeX or this template, but you must turn your homework in as
% a typeset PDF somehow.
%
% How to use:
%    1. Update your information in section "A" below
%    2. Write your answers in section "B" below. Precede answers for all 
%       parts of a question with the command "\question{n}{desc}" where n is
%       the question number and "desc" is a short, one-line description of 
%       the problem. There is no need to restate the problem.
%    3. If a question has multiple parts, precede the answer to part x with the
%       command "\part{x}".
%    4. If a problem asks you to design an algorithm, use the commands
%       \algorithm, \correctness, \runtime to precede your discussion of the 
%       description of the algorithm, its correctness, and its running time, respectively.
%    5. You can include graphics by using the command \includegraphics{FILENAME}
%
\documentclass[11pt]{article}
\usepackage{amsmath,amssymb,amsthm}
\usepackage{graphicx}
\usepackage[margin=1in]{geometry}
\usepackage{fancyhdr}
\setlength{\parindent}{0pt}
\setlength{\parskip}{5pt plus 1pt}
\setlength{\headheight}{13.6pt}
\newcommand\question[2]{\vspace{.25in}\hrule\textbf{#1: #2}\vspace{.5em}\hrule\vspace{.10in}}
\renewcommand\part[1]{\vspace{.10in}\textbf{(#1)}}
\newcommand\algorithm{\vspace{.10in}\textbf{Algorithm: }}
\newcommand\correctness{\vspace{.10in}\textbf{Correctness: }}
\newcommand\runtime{\vspace{.10in}\textbf{Running time: }}
\pagestyle{fancyplain}
\lhead{\textbf{\NAME\ (\ANDREWID)}}
\chead{\textbf{HW\HWNUM}}
\rhead{02-713, \today}
\begin{document}\raggedright
%Section A==============Change the values below to match your information==================
\newcommand\NAME{Carl Kingsford}  % your name
\newcommand\ANDREWID{ckingsf}     % your andrew id
\newcommand\HWNUM{1}              % the homework number
%Section B==============Put your answers to the questions below here=======================

% no need to restate the problem --- the graders know which problem is which,
% but replacing "The First Problem" with a short phrase will help you remember
% which problem this is when you read over your homeworks to study.

\question{1}{The First Problem} 
(a)

\[(s \land e)\lor(m \land e)\lor(\lnot s\land \lnot m\land e)\]
(b)

(c)

\begin{align*}
    p\leftrightarrow q &\equiv (p\to q)\land (q\to p) \\
                       &\equiv (\lnot p \lor q)\land (\lnot q \lor p) \\
                       &\equiv ((\lnot p \lor q)\land \lnot q)\lor((\lnot p \lor q)\land p)\\
                       &\equiv ((\lnot p\land \lnot q)\lor(q\land \lnot q)) \lor ((\lnot p \land p)\lor (q\land p))\\
                       &\equiv (\lnot p\land \lnot q) \lor (q\land p) \\ 
                       &\equiv (p \land q) \lor (\lnot p\land \lnot q)
\end{align*}


\question{2}{The second problem}
To prove \[(p\to q)\land (p\to r)\to (p \to (q\land r))\]

\question{3}{The third problem}
\begin{align*}
    (P\to Q) \lor (Q \to R) &\equiv (\lnot P \lor Q) \lor (\lnot Q \lor R) \\ 
                            &\equiv \lnot P\lor ((Q\lor \lnot Q)\lor R) \\
                            &\equiv \lnot P\lor (R\lor (Q\lor \lnot Q)) \\
                            &\equiv (\lnot P\lor R)\lor (Q \lor\lnot Q) \\
                            &\equiv (\lnot P\lor R) \lor \mathrm{True}\\
                            &\equiv \mathrm{True}
\end{align*}

\question{7}{}
\[s_0 = a_0 + b_0\]
\[s_1 = (a_1 + b_{1}) + a_0b_0\]
\begin{align*}
    s_2 &= (a = (11)_2 \land b \neq (00)_2)\lor (b = (11)_2 \land a \neq (00)_2) \\
        &= a_1a_0\overline{b_0}\overline{b_1} + b_1b_0\overline{a_0}\overline{a_1} + \overline{a_1a_0\overline{b_0}\overline{b_1} + b_1b_0\overline{a_0}\overline{a_1}}
\end{align*}


\question{8}{}
$x\bar{y}$

$x\overline{y}\overline{z}$

\begin{align*}
    (x \land y\land z) \lor (\lnot x \land \lnot y \land \lnot z) &= xyz \lor ((1-x)(1-y)(1-z))\\
    &= xyz + (1-x)(1-y)(1-z) - xyz(1-x)(1-y)(1-z) \\
    &= xy+yz+zx -x -y -z + 1\\
    &= xy + (yz-1) + (zx-1) + (1-x) + (1-y) + (1 - z) \\ 
    &= xy + \overline{yz} + \overline{zx} + \bar{x} + \bar{y} + \bar{z}
\end{align*}

\[a + b + c + d\]

\begin{align*}
(a + b + c + d) \lor (\bar{a} + \bar{b} +\bar{c} +\bar{d}) &= (a + b + c + d) + (\bar{a} + \bar{b} +\bar{c} +\bar{d}) - (a + b + c + d)(\bar{a} + \bar{b} +\bar{c} +\bar{d})\\
\end{align*}

\end{document}

\documentclass{article}

\usepackage{fancyhdr}
\usepackage{extramarks}
\usepackage{amsmath}
\usepackage{amsthm}
\usepackage{amsfonts}
\usepackage{tikz}
\usepackage[plain]{algorithm}
\usepackage{algpseudocode}

\usetikzlibrary{automata,positioning}

%
% Basic Document Settings
%

\topmargin=-0.45in
\evensidemargin=0in
\oddsidemargin=0in
\textwidth=6.5in
\textheight=9.0in
\headsep=0.25in

\linespread{1.1}

\pagestyle{fancy}
\lhead{\hmwkAuthorName}
\chead{\hmwkClass\ (\hmwkClassInstructor\ \hmwkClassTime): \hmwkTitle}
\rhead{\firstxmark}
\lfoot{\lastxmark}
\cfoot{\thepage}

\renewcommand\headrulewidth{0.4pt}
\renewcommand\footrulewidth{0.4pt}

\setlength\parindent{0pt}

%
% Create Problem Sections
%

\newcommand{\enterProblemHeader}[1]{
    \nobreak\extramarks{}{Problem \arabic{#1} continued on next page\ldots}\nobreak{}
    \nobreak\extramarks{Problem \arabic{#1} (continued)}{Problem \arabic{#1} continued on next page\ldots}\nobreak{}
}

\newcommand{\exitProblemHeader}[1]{
    \nobreak\extramarks{Problem \arabic{#1} (continued)}{Problem \arabic{#1} continued on next page\ldots}\nobreak{}
    \stepcounter{#1}
    \nobreak\extramarks{Problem \arabic{#1}}{}\nobreak{}
}

\setcounter{secnumdepth}{0}
\newcounter{partCounter}
\newcounter{homeworkProblemCounter}
\setcounter{homeworkProblemCounter}{1}
\nobreak\extramarks{Problem \arabic{homeworkProblemCounter}}{}\nobreak{}

%
% Homework Problem Environment
%
% This environment takes an optional argument. When given, it will adjust the
% problem counter. This is useful for when the problems given for your
% assignment aren't sequential. See the last 3 problems of this template for an
% example.
%
\newenvironment{homeworkProblem}[1][-1]{
    \ifnum#1>0
        \setcounter{homeworkProblemCounter}{#1}
    \fi
    \section{Problem \arabic{homeworkProblemCounter}}
    \setcounter{partCounter}{1}
    \enterProblemHeader{homeworkProblemCounter}
}{
    \exitProblemHeader{homeworkProblemCounter}
}

%
% Homework Details
%   - Title
%   - Due date
%   - Class
%   - Section/Time
%   - Instructor
%   - Author
%

\newcommand{\hmwkTitle}{Homework\ \#2}
\newcommand{\hmwkDueDate}{February 12, 2014}
\newcommand{\hmwkClass}{Calculus}
\newcommand{\hmwkClassTime}{Section A}
\newcommand{\hmwkClassInstructor}{Professor Isaac Newton}
\newcommand{\hmwkAuthorName}{\textbf{Josh Davis} \and \textbf{Davis Josh}}

%
% Title Page
%

\title{
    \vspace{2in}
    \textmd{\textbf{\hmwkClass:\ \hmwkTitle}}\\
    \normalsize\vspace{0.1in}\small{Due\ on\ \hmwkDueDate\ at 3:10pm}\\
    \vspace{0.1in}\large{\textit{\hmwkClassInstructor\ \hmwkClassTime}}
    \vspace{3in}
}

\author{\hmwkAuthorName}
\date{}

\renewcommand{\part}[1]{\textbf{\large Part \Alph{partCounter}}\stepcounter{partCounter}\\}

%
% Various Helper Commands
%

% Useful for algorithms
\newcommand{\alg}[1]{\textsc{\bfseries \footnotesize #1}}

% For derivatives
\newcommand{\deriv}[1]{\frac{\mathrm{d}}{\mathrm{d}x} (#1)}

% For partial derivatives
\newcommand{\pderiv}[2]{\frac{\partial}{\partial #1} (#2)}

% Integral dx
\newcommand{\dx}{\mathrm{d}x}

% Alias for the Solution section header
\newcommand{\solution}{\textbf{\large Solution}}

% Probability commands: Expectation, Variance, Covariance, Bias
\newcommand{\E}{\mathrm{E}}
\newcommand{\Var}{\mathrm{Var}}
\newcommand{\Cov}{\mathrm{Cov}}
\newcommand{\Bias}{\mathrm{Bias}}

\begin{document}

\maketitle
\pagebreak

\begin{homeworkProblem}
We choose a number from the set $\{1,2,\cdots,100\}$ uniformly at random and denote
this number by $X$. For each of the following choices decide whether the two events in question
are independent or not.

\begin{enumerate}
    \item $A = \{X \text{ is even}\}$, $B = \{X \text{ is divisible by 5}\}$
    \item $C = \{x \text{ has two digits}\}$, $D = \{ X \text{ is divisible by 3}\}$
    \item $E = \{X \text{ is prime}\}$, $F = \{X \text{ has a digit 5}\}$
\end{enumerate}
\solution 
\\\\
Recall the definition of independency: if $X$ and $Y$ are two independent variables, then
$P(X\cap Y) = P(X)\cdot P(Y)$.
\\\\
From there, we have $P(A) = 0.5, P(B) = 0.2, P(A\cap B) = \frac{\#\{10,20,\cdots, 100\}}{100} = \frac{10}{100} = \frac{1}{10} = 0.1$
We have 
\[
    P(A)\cdot P(B) = 0.5 \times 0.2 = 0.1 = P(A\cap B)
\]
Thus $A$ and $B$ are independent.\\
\\
Likewise, we compute \[P(C) = \frac{9}{10}\]
and \[P(D) = \frac{\left[\frac{100}{3}\right]}{100} = \frac{34}{100}\]
and \[P(C\cap D) = \frac{\#\{12,15,18,\cdots, 99\}}{100} = \frac{31}{100}\]
Since
\[P(C)\cdot P(D) = \frac{297}{1000}\neq \frac{31}{100} = P(C\cap D)\]
$C$ is not independent from $D$.
\\\\
There are $25$ primes less than $100$, and we know that
\[F = \{5, 15, 25, 35, 45, 50,\cdots 59, 65, 75, 85, 95\}\]
Thus,
\[\#F = 5 + 10 + 4 = 19\]
We have
\[P(E) = \frac{1}{4}, P(F) = \frac{19}{100}\]
\end{homeworkProblem}

\pagebreak

\begin{homeworkProblem}
Suppose there are two student assistants working as typists in the main office of the
Statistics \& Applied Probability Department at UCSB. The number of typos per page made
by student assistant $A$ is a Poisson random variable with parameter $\lambda_A = 1$. The number of
typos per page made by student assistant $B$ is also a Poisson random variable with an average
of 10 typos per page. One of the professors in the department asks one of the students to type up a letter. From
experience, this work will be done with 1/3 probability by student A and with 2/3 probability
by student B.
\\\\
(a) What is the probability that the typewritten letter will contain exactly one typo?

(b) It turns out that the typewritten letter does not contain any typos. Given this informa
tion, what is the probability that student B typewrote this letter?
\\\\
\solution
\\\\
(a)
Let $T_A, T_B$ be the number of typos made by each assistant in each page respectively.
First off we determine the Possion variable of student $B$, which is 
\[
    P(T_B = k) = \frac{\lambda_B^k\exp(-\lambda_B)}{k!},
    P(T_A = k) = \frac{1^k\exp(-1)}{k!}
\]
where $\lambda_B = 10$.
Thus by \textbf{total probability}, 
\begin{align*}
P(\text{exactly one typo}) &= P(\text{exactly one typo}|\text{student A selected})P(\text{student A selected}) \\ 
                           &+ P(\text{exactly one typo}|\text{student B selected})P(\text{student B selected}) \\
                           &= \frac{1}{3}P(T_A = 1|\text{student A selected}) + \frac{2}{3}P(T_B = 1|\text{ student B selected}) \\
                           &= \frac{1}{3}P\left(T_A = 1\right) + \frac{2}{3}P(T_B = 1) \\
                           &= \frac{1}{3}e^{-1} + \frac{2}{3}10e^{-10} \approx 12.3\%
\end{align*}

(b)
By the Bayes formula, 
\begin{align*}
    P(\text{student B } | \text{ no typo}) &= P(\text{no typo } | \text{ student B})\cdot \frac{P(\text{student B})}{P(\text{no typo})} \\
                                           &= \frac{2}{3}\frac{e^{-10}}{P(T_A = 0 | \text{student A})P(\text{student A}) + P(T_B = 0| \text{student B})P(\text{student B}) }\\
                                           &= \frac{2}{3}\frac{e^{-10}}{1/3\cdot e^{-1} + 2/3\cdot e^{-10}} \\
                                           &\approx 0.2\%
\end{align*}

\end{homeworkProblem}

\pagebreak

\begin{homeworkProblem}
    Suppose you are rolling a fair die $600$ times independently. Let $X$ count the number
of sixes that appear.
\\\\
(a) What type of random variable is $X$? Specify all parameters needed to characterize $X$ as
well as the state space $S_X$ of $X$.

(b) Find the probability that you observe the number 6 at most 100 times.

(c) Use a famous limit theorem (which one?) to show why the probability in (b) can be
approximated by the value $\frac{1}{2}$
\\\\
\solution 

(a)
We observe that $X$ is Binomial distribution whose PMF is 
\[
    P(X = k) = \binom{600}{k}\left(\frac{5}{6}\right)^{600-k}\left(\frac{1}{6}\right)^{k}
\]
where $S_X = \{0,1,2,\cdots,600\}$ 
\\\\
(b)
We need to compute 
\begin{align*}
    P(X\le 100) &= \sum_{k=0}^{100}P(X=k) \\
                &= \sum_{k=0}^{100}\binom{600}{k}\left(\frac{5}{6}\right)^{600-k}\left(\frac{1}{6}\right)^{k}
\end{align*}

From Wolfram Alpha, the summation above approximately equals to the numerical value $0.53$

(c)

\end{homeworkProblem}

\end{document}
\documentclass{article}

\usepackage{fancyhdr}
\usepackage{extramarks}
\usepackage{amsmath}
\usepackage{amsthm}
\usepackage{amsfonts}
\usepackage{tikz}
\usepackage[plain]{algorithm}
\usepackage{algpseudocode}
\usepackage{mathtools}

\usepackage[utf8]{inputenc}

% Default fixed font does not support bold face
\DeclareFixedFont{\ttb}{T1}{txtt}{bx}{n}{12} % for bold
\DeclareFixedFont{\ttm}{T1}{txtt}{m}{n}{12}  % for normal

% Custom colors
\usepackage{color}
\definecolor{deepblue}{rgb}{0,0,0.5}
\definecolor{deepred}{rgb}{0.6,0,0}
\definecolor{deepgreen}{rgb}{0,0.5,0}

\usepackage{listings}

% Python style for highlighting
\newcommand\pythonstyle{\lstset{
language=Python,
basicstyle=\ttm,
morekeywords={self},              % Add keywords here
keywordstyle=\ttb\color{deepblue},
emph={MyClass,__init__},          % Custom highlighting
emphstyle=\ttb\color{deepred},    % Custom highlighting style
stringstyle=\color{deepgreen},
frame=tb,                         % Any extra options here
showstringspaces=false
}}


% Python environment
\lstnewenvironment{python}[1][]
{
\pythonstyle
\lstset{#1}
}
{}

\usetikzlibrary{automata,positioning}

%
% Basic Document Settings
%

\topmargin=-0.45in
\evensidemargin=0in
\oddsidemargin=0in
\textwidth=6.5in
\textheight=9.0in
\headsep=0.25in

\linespread{1.1}

\pagestyle{fancy}
\lhead{\hmwkAuthorName}
\chead{\hmwkClass\ (\hmwkClassInstructor\ \hmwkClassTime): \hmwkTitle}
\rhead{\firstxmark}
\lfoot{\lastxmark}
\cfoot{\thepage}

\renewcommand\headrulewidth{0.4pt}
\renewcommand\footrulewidth{0.4pt}

\setlength\parindent{0pt}

%
% Create Problem Sections
%

\newcommand{\enterProblemHeader}[1]{
    \nobreak\extramarks{}{Problem \arabic{#1} continued on next page\ldots}\nobreak{}
    \nobreak\extramarks{Problem \arabic{#1} (continued)}{Problem \arabic{#1} continued on next page\ldots}\nobreak{}
}

\newcommand{\exitProblemHeader}[1]{
    \nobreak\extramarks{Problem \arabic{#1} (continued)}{Problem \arabic{#1} continued on next page\ldots}\nobreak{}
    \stepcounter{#1}
    \nobreak\extramarks{Problem \arabic{#1}}{}\nobreak{}
}

\setcounter{secnumdepth}{0}
\newcounter{partCounter}
\newcounter{homeworkProblemCounter}
\setcounter{homeworkProblemCounter}{1}
\nobreak\extramarks{Problem \arabic{homeworkProblemCounter}}{}\nobreak{}

%
% Homework Problem Environment
%
% This environment takes an optional argument. When given, it will adjust the
% problem counter. This is useful for when the problems given for your
% assignment aren't sequential. See the last 3 problems of this template for an
% example.
%
\newenvironment{homeworkProblem}[1][-1]{
    \ifnum#1>0
        \setcounter{homeworkProblemCounter}{#1}
    \fi
    \section{Problem \arabic{homeworkProblemCounter}}
    \setcounter{partCounter}{1}
    \enterProblemHeader{homeworkProblemCounter}
}{
    \exitProblemHeader{homeworkProblemCounter}
}

%
% Homework Details
%   - Title
%   - Due date
%   - Class
%   - Section/Time
%   - Instructor
%   - Author
%

\newcommand{\hmwkTitle}{Homework\ \#3}
\newcommand{\hmwkDueDate}{\today}
\newcommand{\hmwkClass}{Discrete Math}
\newcommand{\hmwkClassTime}{Section A}
\newcommand{\hmwkClassInstructor}{Professor J}
\newcommand{\hmwkAuthorName}{\textbf{V} \and \textbf{U}}

%
% Title Page
%

\title{
    \vspace{2in}
    \textmd{\textbf{\hmwkClass:\ \hmwkTitle}}\\
    \normalsize\vspace{0.1in}\small{Due\ on\ \hmwkDueDate\ at 3:10pm}\\
    \vspace{0.1in}\large{\textit{\hmwkClassInstructor\ \hmwkClassTime}}
    \vspace{3in}
}

\author{\hmwkAuthorName}
\date{}

\renewcommand{\part}[1]{\textbf{\large Part \Alph{partCounter}}\stepcounter{partCounter}\\}

%
% Various Helper Commands
%

% Useful for algorithms
\newcommand{\alg}[1]{\textsc{\bfseries \footnotesize #1}}

% For derivatives
\newcommand{\deriv}[1]{\frac{\mathrm{d}}{\mathrm{d}x} (#1)}

% For partial derivatives
\newcommand{\pderiv}[2]{\frac{\partial}{\partial #1} (#2)}

% Integral dx
\newcommand{\dx}{\mathrm{d}x}

% Alias for the Solution section header
\newcommand{\solution}{\textbf{\large Solution}}

% Probability commands: Expectation, Variance, Covariance, Bias
\newcommand{\E}{\mathrm{E}}
\newcommand{\Var}{\mathrm{Var}}
\newcommand{\Cov}{\mathrm{Cov}}
\newcommand{\Bias}{\mathrm{Bias}}

\begin{document}

\maketitle

\pagebreak

\begin{homeworkProblem}
    
	Use the Euclidean algorithm to calculate gcd(102,70).
	Use the extended Euclidean algorithm to write gcd(663,234) as an integer linear combination of 663 and 234.\\
	\\
    \textbf{Solution.}
	Euclidean algorithm can be described as followed:

	\begin{equation}
		\gcd(a, b) :=
		\begin{cases*}
		  a & if $b = 0$ \\
		  \gcd(b, a\mod b)       & otherwise
		\end{cases*}
	\end{equation}

	Thus 
	\begin{align*}\gcd(102, 70) &= \gcd(20, 32) \\
		&= \gcd(32, 6) \\
		&= \gcd(6,2)\\ 
		&= \gcd(2, 0)\\
		&= 2
	\end{align*}

Running the following python code, we have
	\begin{python}
		def extended_gcd(a, b):
			if a < b:
				a, b = b, a

			old_r, r = a, b
			old_s, s = 1, 0
			old_t, t = 0, 1

			while r != 0:
				quotient = old_r // r
				old_r, r = r, old_r - quotient * r
				old_s, s = s, old_s - quotient * s
				old_t, t = t, old_t - quotient * t
			
			print("Bézout coefficients:", (old_s, old_t))
			print("greatest common divisor:", old_r)
			print("quotients by the gcd:", (t, s))
		
		extended_gcd(663, 234)
	\end{python}
	
	Bézout coefficients: (-1, 3)\\
	greatest common divisor: 39\\
	quotients by the gcd: (-17, 6)

	Therefore, $-1 \times 663 + 3 \times 234 = \gcd(663, 234) = 39$
    \qed
\end{homeworkProblem}

\pagebreak

\begin{homeworkProblem}
	Prove that a number is divisible by $3$ if and only if the sum of its digits is divisible by $3$.\\
	\\
	\textbf{Solution.} \\
	Let $n \in \mathbb{N}$, and let $n = \sum_{k = 0}^{m} a_k10^k$ be the decimal representation of $n$.
	We need to show
	\[n \equiv 0 \mod 3 \iff \sum_{k = 0}^{m} a_k \equiv 0 \mod 3\]
	In fact,
	\begin{align*}
		n &= \sum_{k=0}^{m} a_k10^k \\
		  &= \sum_{k=0}^{m} a_k(3\times 3 + 1)^k \\
		  &\equiv \sum_{k=0}^{m} a_k \mod 3
	\end{align*}

	which is sufficient.
	\qed
\end{homeworkProblem}

\begin{homeworkProblem}
	Prove that all numbers in the sequence
		\[1007,10017,100117,1001117,\cdots\]
	are divisible by 53.\\
	\\
	\textbf{Solution.}

	The numbers in this sequence can be formulated as

	\[a_n = 100\times 10^{n} + \sum_{k=0}^{n-1}10^{k} + 6\]
	\[\iff 10(a_n -6) + 6 + 1 = 100 \times 10^{n+1} + \sum_{k=0}^{n}10^k + 6 = a_{n+1}\]
	\[\iff a_{n + 1} = 10a_n - 53\]

	Thus $a_{n + 1} = 10a_n \mod 53$. $a_0 = 1007 = 19\times 53 \equiv 0 \mod 53$.
	By induction, $a_{n+1} = 0 \mod 53,\forall n\in \mathbb{N}$.
	\qed
\end{homeworkProblem}

\pagebreak

\begin{homeworkProblem}
	4.A robot walks around a two-dimensional grid. It starts out at (2,0) and is allowed to take four different types of steps as:
	\begin{enumerate}
		\item $(+2,-1)$
		\item $(+1,-2)$
		\item $(+1,+4)$
		\item $(-3,0)$
	\end{enumerate}
Prove that this robot can never reach $(0,-1)$.\\

\textbf{Solution.}

Note that the moves of the robot satisfying commutative. 

Let $a, b, c, d$ be the number of moves of $1, 2, 3, 4$.
we have system of equation
\begin{equation}
    \begin{cases}
      2 + 2a + b + c -3d &= 0\\
	  -a - 2b + 4c &= -1
    \end{cases}\,
\end{equation}

Act $\mod 3$ on this system, we have

\begin{equation}
    \begin{cases}
      -a + b + c &= 1\\
	  -a + b + c &= -1
    \end{cases}\,
\end{equation}

which is contradiction. Thus the original system has no solution at all, which implies that the robot has no way to reach $(0, -1)$.
\qed
\end{homeworkProblem}

\pagebreak

\begin{homeworkProblem}
	NIM is a famous game in which two players take turns removing items from a pile of $n$ items. For every turn, the player can remove one, two, or three items at a time. 
	The player removing the last item loses. Prove that if each player plays the best strategy possible, the first player wins if $n \not \equiv 1 (\mod 4)$ and the second player wins if $n \equiv 1 (\mod 4)$.
	(For your interest, refer to the general NIM game at this link).
\\\\
	\textbf{Solution.}
	If $n \equiv 1 \mod 4$, the first player can only change $n$ so that $n$ divided by $4$ remains $1 - 1, 1 - 2 \text{ or } 1 - 3$ which are $0, 3, 2$ excluding $1$.
	Then in the second turn, the second player can pick a number of items $n$ so that $n \equiv 1 \mod 4$ again. At the end of the game, it will always be the first player taking the last item.

	Therefore, if $n \equiv 1 \mod 4$, the second player wins. Since the NIM game has no draw, under optimal strategy, in any other cases for $n$, the first player wins.
	\qed
\end{homeworkProblem}

\pagebreak

%
% Non sequential homework problems
%

\begin{homeworkProblem}
	Find all solutions, if any, to the system:
	\begin{equation}
		\begin{cases}
		  x \equiv 5 \mod 6\\
		  x \equiv 3 \mod 10\\
		  x \equiv 8 \mod 15
		\end{cases}\,
	\end{equation}

	\textbf{Solution.}

	By the divisibility of primes, the system can be reduced as
	\begin{equation}
		\begin{cases}
		  x \equiv 5 \mod 2\\
		  x \equiv 5 \mod 3 \\
		  \\
		  x \equiv 3 \mod 2\\
		  x \equiv 3 \mod 5\\
		  \\
		  x \equiv 8 \mod 3\\
		  x \equiv 8 \mod 5
		\end{cases}\,
	\end{equation}

	After Simplification.
	\begin{equation}
		\begin{cases}
		  x \equiv 1 \mod 2\\
		  x \equiv 2 \mod 3\\
		  x \equiv 3 \mod 5
		\end{cases}\,
	\end{equation}
    By Chinese remainder theorem, 
	It has an unique solution in $\mathbb{Z}/30\mathbb{Z}$. After enumeration, 
	\[x \equiv 23 \mod 30\]
	\qed
\end{homeworkProblem}

% Continue counting to 19
\begin{homeworkProblem}
    Show with the help of Fermat's little theorem that if $n$ is a positive integer, then $42|n^7 - n$. \\ \\
	\textbf{Solution.}\\
	By divisibility of prime, we need only to show $2 | n^7 - n$, $3 | n^7 - n$, and $7 | n^7 - n$.\\
	By Fermat's Little Theorem ($n^p \equiv n \mod p, \forall p\in \mathrm{prime}$), \[7 | n^7 - n\].
	Likewise,
	\[n^7 = n^2 n^2 n^2 n \equiv n n n n = n^2 n^2 \equiv nn = n^2 = n \mod 2\]
	Also,
	\[n^7 = n^3 n^3 n \equiv n n n = n^3 \equiv n \mod 3\]
	Thus the proof is as desired.
	\qed
\end{homeworkProblem}

\end{document}
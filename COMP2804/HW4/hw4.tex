%=======================02-713 LaTeX template, following the 15-210 template==================
%
% You don't need to use LaTeX or this template, but you must turn your homework in as
% a typeset PDF somehow.
%
% How to use:
%    1. Update your information in section "A" below
%    2. Write your answers in section "B" below. Precede answers for all 
%       parts of a question with the command "\question{n}{desc}" where n is
%       the question number and "desc" is a short, one-line description of 
%       the problem. There is no need to restate the problem.
%    3. If a question has multiple parts, precede the answer to part x with the
%       command "\part{x}".
%    4. If a problem asks you to design an algorithm, use the commands
%       \algorithm, \correctness, \runtime to precede your discussion of the 
%       description of the algorithm, its correctness, and its running time, respectively.
%    5. You can include graphics by using the command \includegraphics{FILENAME}
%
\documentclass[11pt]{article}
\usepackage{amsmath,amssymb,amsthm}
\usepackage{graphicx}
\usepackage[margin=1in]{geometry}
\usepackage{fancyhdr}
\usepackage{cancel}

\setlength{\parindent}{0pt}
\setlength{\parskip}{5pt plus 1pt}
\setlength{\headheight}{13.6pt}
\newcommand\question[2]{\vspace{.25in}\hrule\textbf{#1: #2}\vspace{.5em}\hrule\vspace{.10in}}
\renewcommand\part[1]{\vspace{.10in}\textbf{(#1)}}
\newcommand\algorithm{\vspace{.10in}\textbf{Algorithm: }}
\newcommand\correctness{\vspace{.10in}\textbf{Correctness: }}
\newcommand\runtime{\vspace{.10in}\textbf{Running time: }}
\newcommand{\solution}{\vspace{.10in}\textbf{Solution: }}
\pagestyle{fancyplain}
\lhead{\textbf{\NAME\ (\ANDREWID)}}
\chead{\textbf{HW\HWNUM}}
\rhead{02-713, \today}
\begin{document}\raggedright
%Section A==============Change the values below to match your information==================
\newcommand\NAME{Carl Kingsford}  % your name
\newcommand\ANDREWID{ckingsf}     % your andrew id
\newcommand\HWNUM{1}              % the homework number
%Section B==============Put your answers to the questions below here=======================

% no need to restate the problem --- the graders know which problem is which,
% but replacing "The First Problem" with a short phrase will help you remember
% which problem this is when you read over your homeworks to study.
\question{2}{}
You and your friend decide to play a dice game. You each know a little about
probability - for instance, you know that if you roll 2 dice, the most likely number to come up is 7.
You suggest the following game. You keep rolling the dice one of the following two events happens:
\begin{itemize}
    \item If the roll is 3 or 11, you friend wins.
    \item if a 7 comes twice before a 3 or 11 is rolled, you win.
\end{itemize}
Each roll is mutually independent.

\part{a} What is the probability that you win and what is the probability that your friend wins?

\solution

Let $FW$ be the event that my friend wins, and $MW$ be the event that I win.

\begin{align*}
        P(\lnot(3 \lor 11))   &= 1 - P(3 \lor 11) \\
           &= 1 - P(A = 1 \land B = 2) - P(A = 2 \land B = 1) - P(A = 5 \land B = 6) - P(A = 6 \land B = 5) \\
           &= 1 - \frac{1}{36} - \frac{1}{36} - \frac{1}{36} - \frac{1}{36} \\
           &= 1 - \frac{4}{36}  = \frac{32}{36}
\end{align*}

And 
\begin{align*}
    P(\lnot 3 \lnot 7 \lnot 11) = \frac{8}{9} - \frac{1}{6} = \frac{13}{18}
\end{align*}

Thus,
\begin{align*}
    P(FW) &= P\left( 3\  or\  11\right)  \cdot P\left( FW|3\  or\  11\right)  +P\left( 7\right)  P\left( FW|7\right) + P(others) P(FW|others) \\
    P\left( FW|7\right)  &= P\left( 7\right)  P\left( FW|77\right)  +P\left( 3\  or\  11\right)  P\left( FW|7|3or11\right)  +P\left( other\right)  P\left( FW|7\right)
\end{align*}

Thus,

\begin{equation*}
    \begin{cases}
        P(FW) = \frac{4}{36} \cdot 1 + \frac{1}{6} \cdot P(FW|7) + \frac{13}{18} \cdot P(FW) \\
        P(FW|7) = \frac{4}{36} \cdot 1 + \frac{13}{18}F(FW|7) \\
    \end{cases}
\end{equation*}
It gives
\begin{equation*}
    \begin{cases}
        P(FW) = \frac{16}{25} \\
        P(FW|7) = \frac{2}{5} \\
    \end{cases}
\end{equation*}
Thus $P(MW) = 1 - \frac{16}{25} = \frac{9}{25}$

\part{2}
You decide to play for money. Every time a 3 or 11 comes up, you pay your friend \$3. Every time a 7 comes up, he pays you \$2. What is your expected winnings per die roll?

\solution

Let $M$ be the random variable of my winnings per die roll.
\begin{equation}
    E(M) = (-3)\frac{4}{36} + 2\frac{6}{36} = 0
\end{equation}

\part{c}
How many dice rolls would you expect there to be before a 3, 7, or 11 is rolled?

\solution

Let $X$ be the random variable of the number of dice rolls before a 3, 7, or 11 is rolled.
We need to calculate $E(X)$. And let $A$ be the number of 3, 7, or 11 is rolled in the first roll.
Thus, 
\begin{equation}
    E\left( X|A\right)  =\begin{cases}1&\text{with probability} \  5/18\\ 1+E(X)&\text{with probability} \  13/18\end{cases} 
\end{equation}
    
Then, we have
\begin{align*}
    E(X) &= E\left( E\left( X|A\right) \right) \\
         &= \frac{5}{18} \times 1 + \frac{13}{18} \times \left(1 + E(X)\right) \\
         &= 1 + \frac{13}{18} E(X) \\
    \frac{5}{18} E(X) &= 1 \\
    E(X) &= \frac{18}{5}
\end{align*}

\part{c} How many dice roll would you expect before someone wins the original game? That is, how many dice rolls would you expect before a 7 is rolled twice or a 3 or 11 is rolled once?

\solution

Let $X$ be the random variable of the number of dice rolls before a 7 is rolled twice or a 3 or 11 is rolled once.
Let $Y$ be the random variable of the number of dice rolls before a 7 is rolled once or a 3 or 11 is rolled once.

We need to calculate $E(X)$. And let $A$ be the value of the first roll.
Thus,
\begin{align*}
    E(X) = E(E(X|A)) &= E(X|A=3 \lor A=11) \times P(A = 3 \lor A=7) + E(X|A = 7) \times P(A=7) \\
                     &+ E(X| A \neq 3 \land A \neq 11 \land A \neq 7) \times P(A \neq 3 \land A \neq 11 \land A \neq 7) \\
                     &= 1 \times P(A = 3 \lor A=11) + (1 + E(Y)) \times P(A=7) + (1 + E(X)) \times P(A \neq 3 \land A \neq 11 \land A \neq 7) \\
\end{align*}

And

\begin{align*}
    E(Y) = E(E(Y|A)) &= E(Y|A=3 \lor A=11) \times P(A = 3 \lor A=7) + E(Y|A = 7) \times P(A=7) \\ 
                     &+ E(Y| A \neq 3 \land A \neq 11 \land A \neq 7) \times P(A \neq 3 \land A \neq 11 \land A \neq 7) \\
                     &= 1 \times P(A = 3 \lor A=11) + 1 \times P(A=7) + (1 + E(Y)) \times P(A \neq 3 \land A \neq 11 \land A \neq 7) \\
\end{align*}

Note that $P(A = 3 \lor A=11) = \frac{1}{9}$, $P(A=7) = \frac{1}{6}$, $P(A \neq 3 \land A \neq 11 \land A \neq 7) = \frac{13}{18}$.

Let $x=E(X), y=E(Y)$, we have a system of equations:

\begin{align*}
    x &= 1 \times \frac{1}{9} + (1 + y) \times \frac{1}{6} + (1 + x) \times \frac{13}{18} \\
    y &= 1 \times \frac{1}{9} + 1 \times \frac{1}{6} + (1 + y) \times \frac{13}{18} \\
\end{align*}

solve this we have $x = \frac{144}{25}, y = \frac{18}{5}$.

\question{4}{}
You are in a class of $200$ people. Let $X$ be the number of different birthdays among these 200 people (assuming no one was born on February 29, i.e., on a leap year). Determine the expected value $E(X)$ of $X$.
Hint: Use indicator random variables.

\solution

Let $X_i$ be the indicator random variable of the date $i$ is the birthday of at least one person in the class.
Then, we have

\begin{equation}
    X = \sum_{i=1}^{365} X_i
\end{equation}

And

\begin{equation}
    E(X) = E\left(\sum_{i=1}^{365} X_i\right) = \sum_{i=1}^{365} E(X_i) = \sum_{i=1}^{365} P(X_i = 1)
\end{equation}

Note that $P(X_i = 1) = 1 - P(X_i = 0) = 1 - \left(\frac{364}{365}\right)^{200}$.

Thus,

\begin{equation}
    E(X) = 365 \times \left(1 - \left(\frac{364}{365}\right)^{200}\right)
\end{equation}

\question{5}{}
Consider a fair 6-sided die. 

\part{a} Roll the die twice. What is the expected value of the highest number?

\solution

Let $X, Y$ be the random variables of the first and second roll, respectively.
Let $Z = \max(X, Y)$

Then, we have

\begin{align*}
    E(Z) &= \sum_{i=1}^{6}\sum_{j=1}^{6}\max(i,j)P(X=i)P(Y=j) \\
         &= 2\sum_{i \le j}jP(X=i)P(Y=j) - \sum_{i=1}^{6}iP(X=i)P(Y=i) \\
         &= 2\sum_{j=1}^{6}\sum_{i=1}^{j} jP(X=i)P(Y=j) - \frac{7} {12} \\
         &= 2\sum_{j=1}^{6} jP(Y=j) \sum_{i=1}^{j} P(X=i) - \frac{7} {12}\\
         &= 2\sum_{j=1}^{6} jP(Y=j) \sum_{i=1}^{j} \frac{1}{6}- \frac{7} {12} \\
         &= 2\sum_{j=1}^{6} jP(Y=j) \frac{j}{6} - \frac{7} {12}\\
         &= \frac{1}{18}\sum_{j=1}^{6} j^2 - \frac{7} {12}\\
         &= \frac{1}{3} \times \frac{6 \times 7 \times 13}{6} - \frac{7} {12}\\
         &= \frac{161}{36}
\end{align*}

\part{b} Roll the die once. If the number is $ > 3$ keep that number. Otherwise roll the die again and keep the highest of the two rolls. What is the expected value?

\solution

Let $Y$ be the random variable of the number of the first roll, and let $X$ be final number we keep.

Then, we have

\begin{align*}
    E(X) &= E(E(X|Y)) \\
         &= E(X|Y>3)P(Y>3) + E(X|Y\le 3)P(Y\le 3) \\
\end{align*}

Note that $E(X|Y>3) = E(Y|Y>3) = \frac{4+5+6}{3} = 5$, and 
\begin{align*}
    E(X|Y \le 3) &= E(\max(Y, Z) | Y \le 3)
\end{align*}
    
    where $Z$ is the second roll. Note that $Z$ is independent of $Y$. Thus,

\begin{align*}
    E(X|Y \le 3) &= E(\max(Y, Z) | Y \le 3) \\
                 &= \sum_{i=1}^{3} \sum_{j=1}^{6} \max(i, j) P(Y=i) P(Z=j) \\
                 &= \sum_{i=1}^{3} \sum_{j=1}^{3} \max(i,j) P(Y=i) P(Z=j) + \sum_{i=1}^{3} \sum_{j=4}^{6} j P(Y=i) P(Z=j) \\
                 &= \sum_{1\le i \le j \le 3} j P(Y=i) P(Z=j) + \sum_{1\le j < i \le 3} i P(Y = i)P(Z = j) + \sum_{i=1}^{3} \sum_{j=4}^{6} j P(Y=i) P(Z=j) \\
                 &= 1 \cdot P(Y = 1) P(Z = 1) + 2 \cdot P(Y = 1) P(Z = 2) + 2 \cdot P(Y = 2) P(Z = 2) + 3 \cdot P(Y = 1) P(Z = 3) \\
                 &+ 3 \cdot P(Y = 2) P(Z = 3) + 3 \cdot P(Y = 3) P(Z = 3) \\ 
                 &+ 2 \cdot P(Y = 2) P(Z = 1) + 3 \cdot P(Y = 3) P(Z = 1) + 3 \cdot P(Y = 3) P(Z = 2) \\
                 &+ \sum_{i=1}^{3} \sum_{j=4}^{6} j P(Y=i) P(Z=j) \\ 
\end{align*}

Note that $P(Y = i) = \frac{1}{3}, P(Z = i) = \frac{1}{6}$,

\begin{align*}
    E(X|Y \le 3) &= \frac{1}{18} + \frac{2}{18} + \frac{2}{18} + \frac{3}{18} + \frac{3}{18} + \frac{3}{18} + \frac{2}{18} + \frac{3}{18} + \frac{3}{18} + \sum_{i=1}^{3} \sum_{j=4}^{6} j \frac{1}{18} \\
                 &= \frac{7}{9} + \frac{2}{18} + \frac{3}{18} + \frac{3}{18} +\frac{1}{18} \sum_{i=1}^{3} \sum_{j=4}^{6} j \\
                 &= \frac{11}{9} + \frac{45}{18}\\
                 &= \frac{67}{18}
\end{align*}

Finally, we have

\begin{align*}
    E(X) &= E(X|Y>3)P(Y>3) + E(X|Y\le 3)P(Y\le 3) \\
         &= 5 \times \frac{1}{2} + \frac{67}{18} \times \frac{1}{2} \\
         &= \frac{157}{36} \approx 4.3611
\end{align*}

\part{c} Roll the die once. If the number is $ > 4$ keep that number. Otherwise roll the die again and keep the highest of the two rolls. What is the expected value? What strategy gives us the highest value on average?

\solution

Let $Y$ be the random variable of the number of the first roll, and let $X$ be final number we keep.

Then, we have

\question{6}{}

In this question we will consider bitstrings. A bit is called lonely if it is a $1$ and every adjacent bit is a $0$. A bit is not lonely if it is a $1$ and it is adjacent to at least one other $1$.

\part{a}
Consider a random bitstring of length 10. What is the expected number of lonely bits?

\solution

Let $X_i$ be the indicator random variable of the $i$-th bit is lonely.
Then, we have

\begin{equation}
    X = \sum_{i=1}^{10} X_i 
\end{equation}

And

\begin{equation}
    E(X) = E\left(\sum_{i=1}^{10} X_i\right) = \sum_{i=1}^{10} E(X_i) = \sum_{i=1}^{10} P(X_i = 1)
\end{equation}

Note that for $i = 1$ and $i = 10$, $P(X_i = 1) = \frac{1}{2} \times \frac{1}{2} = \frac{1}{4}$.
While for $i \in [2, 9]$, $P(X_i = 1) = \frac{1}{2} \times \frac{1}{4} = \frac{1}{8}$.

Thus,

\begin{equation}
    E(X) = 2 \times \frac{1}{4} + 8 \times \frac{1}{8} = \frac{3}{2}
\end{equation}

\part{b} We choose a bitstring uniformly at random from all bitstrings of length 5 with exactly three 1's. What is the expected number of lonely bits?

\solution

\begin{align*}
    E(\text{lonely}) &= 3\cdot\frac{1}{10} + 1\cdot\frac{6}{10}+ 0\cdot\frac{3}{10}\\
                     &= \frac{9}{10} \\
\end{align*}

\part{c} what is the expected number of not lonely bits?

\solution

Let $X_i$ be the indicator random variable of the $i$-th bit is \textit{not lonely}.
Then, we have

\begin{equation}
    X = \sum_{i=1}^{10} X_i
\end{equation}

And

\begin{equation}
    E(X) = E\left(\sum_{i=1}^{10} X_i\right) = \sum_{i=1}^{10} E(X_i) = \sum_{i=1}^{10} P(X_i = 1)
\end{equation}

Note that for $i = 1$ and $i = 10$, $P(X_i = 1) = \frac{1}{2} \times \frac{1}{2} = \frac{1}{4}$.
While for $i \in [2, 9]$, $P(X_i = 1) = \frac{1}{2} \times \frac{3}{4} = \frac{3}{8}$.

Thus,

\begin{equation}
    E(X) = 2 \times \frac{1}{4} + 8 \times \frac{3}{8} = \frac{7}{2}
\end{equation}

\part{d} We choose a bitstring uniformly at random from all bitstrings of length 10 with exactly four 1's. What is the expected number of lonely and not lonely bits?

\solution

\begin{align*}
    E(\text{lonely}) &= 4\cdot\frac{\binom{7}{4}}{\binom{10}{4}} + 2 \cdot \frac{\binom{7}{1}\binom{6}{2}}{\binom{10}{4}} + 1 \cdot \frac{\binom{7}{1}\binom{6}{1}}{\binom{10}{4}} + 0 \cdot \frac{\binom{7}{1} + \binom{7}{2}}{\binom{10}{4}} \\
        &= 4\cdot \frac{35}{210} + 2 \cdot \frac{7 \cdot 15}{210} + 1 \cdot \frac{7 \cdot 6}{210} + 0 \cdot \frac{7 + 21}{210} \\
        &= \frac{28}{15}
\end{align*}

\begin{align*}
    E(\text{not lonely}) &= 4 - E(\text{lonely}) \\
                         &= \frac{32}{15}
\end{align*}

\question{7}{} We have a fair, 6-sided die. We roll this die until the sum of all rolls is $\ge 2$. Let X be the number of rolls, and let Y be the sum of all the rolls.

\part{a} What is $E(X)$, use the formula $E(X)=\sum_{\forall k} k\cdot P(X=k)$.

\solution

\begin{align*}
    E(X) &= 1\cdot P(X=1) + 2\cdot P(X=2) + 0 \\
         &= 1\cdot\frac{5}{6} + 2\cdot\left(\frac{1}{6}\cdot 1\right) \\
         &= \frac{7}{6}
\end{align*}

\part{b} What is $E(Y)$, use the formula $E(Y)=\sum_{\forall k} k\cdot P(Y=k)$.

\solution
The expected value is
\begin{equation*}
 \left(2 + 3+4+\cdots +6\right)  \cdot \left( \frac{1}{6} +\frac{1}{36} \right)  +7\cdot \frac{1}{36} =\frac{49}{12}
\end{equation*}

\part{c} Let D be the value of a single die roll. We have seen in class that $E(D) = 3.5$. What is $E(X) \cdot E(D)$?

\solution

\begin{equation*}
    E(X) \cdot E(D) = \frac{7}{6} \cdot 3.5 = \frac{7}{6} \cdot \frac{7}{2} = \frac{49}{12}
\end{equation*}

\part{d} This is an example of Wald's Identity. Wald's Identity tells us that if X is the number of die rolls, and the value of $X$ depends on a stopping condition (which it does in this case), then the expected sum $E(Y) = E(X) \cdot E(D)$. Find $E(X)$ if X is the number of rolls until the sum is $\ge 3$. Then find the corresponding value $E(Y)$ using Wald's Identity.

\solution

Let $X$ be the random variable of the number of rolls until the sum is $\ge 3$.
Then, we have

\begin{align*}
    E(X) &= 1\cdot P(X=1) + 2\cdot P(X=2) + 3\cdot P(X=3) \\ 
         &= 1\cdot \frac{4}{6} + 2\cdot \left(\frac{1}{6}\cdot\frac{5}{6} + \frac{1}{6}\cdot 1\right) + 3\cdot\left(\frac{1}{6}\cdot\frac{1}{6}\cdot 1\right) \\
         &= \frac{49}{36}
\end{align*}

Thus $E(Y) = E(X) \cdot E(D) = \frac{49}{36} \cdot 3.5 = \frac{343}{72}$.

\question{8}{}

If $X$ is a random variable that can take any value n where n is an integer and $n \ge 1$, and if A is an event, then the conditional expected value $E(X|A)$ is defined as
\begin{equation}
    E(X|A) = \sum_{k=1}^{\infty} k\cdot P(X=k|A)
\end{equation}

You roll a fair six-side die repeatedly until you see the number 6. Define the random variable $X$ to be the number of die rolls
(including the last roll where you see 6). We have seen in class that $E(X) = 6$. Let A be the event

\[A = \text{“You do not roll 6 on the first two rolls”.}\]

What is $E(X|A)$?

\solution

\begin{align*}
    E(X|A) &= \sum_{k=1}^{\infty} k\cdot P(X=k|A) \\
           &= \sum_{k=3}^{\infty} k\cdot P(X=k|A) \\
           &= \sum_{k=3}^{\infty} k\cdot \frac{1}{6} \left(\frac{5}{6}\right)^{k-3} \\
           &= 8
\end{align*}
    
\end{document}
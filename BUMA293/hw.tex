%=======================02-713 LaTeX template, following the 15-210 template==================
%
% You don't need to use LaTeX or this template, but you must turn your homework in as
% a typeset PDF somehow.
%
% How to use:
%    1. Update your information in section "A" below
%    2. Write your answers in section "B" below. Precede answers for all 
%       parts of a question with the command "\question{n}{desc}" where n is
%       the question number and "desc" is a short, one-line description of 
%       the problem. There is no need to restate the problem.
%    3. If a question has multiple parts, precede the answer to part x with the
%       command "\part{x}".
%    4. If a problem asks you to design an algorithm, use the commands
%       \algorithm, \correctness, \runtime to precede your discussion of the 
%       description of the algorithm, its correctness, and its running time, respectively.
%    5. You can include graphics by using the command \includegraphics{FILENAME}
%
\documentclass[11pt]{article}
\usepackage{amsmath,amssymb,amsthm}
\usepackage{graphicx}
\usepackage[margin=1in]{geometry}
\usepackage{fancyhdr}
\usepackage{mathtools}
\DeclarePairedDelimiter{\ceil}{\lceil}{\rceil}
\setlength{\parindent}{0pt}
\setlength{\parskip}{5pt plus 1pt}
\setlength{\headheight}{13.6pt}
\newcommand\question[2]{\vspace{.25in}\hrule\textbf{#1: #2}\vspace{.5em}\hrule\vspace{.10in}}
\renewcommand\part[1]{\vspace{.10in}\textbf{(#1)}}
\newcommand\algorithm{\vspace{.10in}\textbf{Algorithm: }}
\newcommand\correctness{\vspace{.10in}\textbf{Correctness: }}
\newcommand\runtime{\vspace{.10in}\textbf{Running time: }}
\newcommand{\sumn}{\sum_{k=1}^{n}}
\pagestyle{fancyplain}
\lhead{\textbf{\NAME\ (\ANDREWID)}}
\chead{\textbf{HW\HWNUM}}
\rhead{02-713, \today}
\begin{document}\raggedright
%Section A==============Change the values below to match your information==================
\newcommand\NAME{Carl Kingsford}  % your name
\newcommand\ANDREWID{ckingsf}     % your andrew id
\newcommand\HWNUM{1}              % the homework number
\newcommand\suminf{\sum_{n=0}^{\infty}}
%Section B==============Put your answers to the questions below here=======================

% no need to restate the problem --- the graders know which problem is which,
% but replacing "The First Problem" with a short phrase will help you remember
% which problem this is when you read over your homeworks to study.

\question{1}{The First Problem}

Let $d = \gcd(x,y)$, $x=x'd, y=y'd$. Then $\gcd(x', y') =1$ Then
\[(x'd)^2 = 2(y'd)^2\implies x'^2 = 2y'^2\]
Then we deduced that $x'$ and $y'$ are even, which contradicts to that fact that $\gcd(x',y')=1$

\question{1.6.2}{}
\begin{align*}
    n^2 + 7n + 12 = (n^2 + n) + (6n + 12)
\end{align*}

\question{1.7}{}
Let $n = 7m^2$, then $7n + 2 = 7(7m^2) + 2 = (7m)^2 + 2$
This satisfies that

\[(7m)^2< (7m)^2 + 2 < (7m)^2 + 14m + 1 = (7m+1)^2\]

\question{2.3.4}{}
\[D_n = \{d | d \text{ divies } n\}\]
\[D_r \cap D_s = \{d | d \text{ divides } r \text{ and } d \text{ divides } s\}\]
Let $m = \gcd(r, s)$
\[D_m = \{d | d \text{ divides } m\}\]
We now have to show that \[D_m \subseteq D_r \cap D_s\].
And
\[D_r \cap D_s \subseteq D_m\]

Let $d \in D_m$, then $d \text{ divides } m$. With $m$ dividing $r$ and $s$, we have $d$ divides $r$ and $s$.
Thus, 
\[D_m \subseteq D_r \cap D_s\]
On the other hand, Let $d \in D_r \cap D_s$, we have $d$ divides both $r$ and $s$, then $d$ divides $m$ by the definition of $m$.
Therefore, \[D_r \cap D_s \subseteq D_m\]
Together, we have proven that 
\[D_m = D_r \cap D_s\]
\pagebreak
\question{2.4}{}
Those subsets can be divided into $n + 1$ groups, and in each group, each subset has $r$ elements for $r = 0, 1, \cdots, n$.
So in total there are
\[\sum_{r=0}^{n}\binom{n}{r}\]
such subsets

\end{document}

